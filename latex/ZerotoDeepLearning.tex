\documentclass[cn,11pt,twocol]{elegantbook}

\usepackage{graphicx}
\usepackage{subfigure}
\usepackage{multirow}
\usepackage{animate}

\lstset{
    columns=fixed,       
    numbers=left,                                        % 在左侧显示行号
    frame=none,                                          % 不显示背景边框
    backgroundcolor=\color[RGB]{245,245,244},            % 设定背景颜色
    keywordstyle=\color[RGB]{40,40,255},                 % 设定关键字颜色
    numberstyle=\footnotesize\color{darkgray},           % 设定行号格式
    commentstyle=\it\color[RGB]{0,96,96},                % 设置代码注释的格式
    stringstyle=\rmfamily\slshape\color[RGB]{128,0,0},   % 设置字符串格式
    showstringspaces=false,                              % 不显示字符串中的空格
    language=Python,                                     % 设置语言
}

\title{深度学习:零基础入门}
\subtitle{从零实现深度学习(Python版)}

% \author{Bingtao Han\& Deans Yu}
\author{Bingtao Han}
\institute{Open Access}
\date{\today}
\version{0.09}

\extrainfo{Victory won\rq t come to us unless we go to it. --- M. Moore}

\logo{dl.jpg}
\cover{cover3.jpg}
% \logo{seulogo.png}
% \cover{cover.jpg}

\begin{document}

\maketitle

\tableofcontents

\mainmatter
\hypersetup{pageanchor=true}


\input{./chapter/perceptron.tex}
\input{./chapter/linear_unit.tex}
\input{./chapter/bp.tex}
\chapter{卷积神经网络}\label{chap:Cnn}

\begin{introduction}
	\item 激活函数:Relu~\ref{Cnn:1}
	\item 全连接网络 VS 卷积网络~\ref{Cnn:2}
	\item 卷积神经网络是啥~\ref{Cnn:3}
	\item 卷积神经网络输出值的计算~\ref{Cnn:4}
	\item 卷积层输出值的计算~\ref{Cnn:5}
	\item Pooling层输出值的计算~\ref{Cnn:6}
	\item 卷积神经网络的训练~\ref{Cnn:7}
	\item 卷积层的训练~\ref{Cnn:8}
	\item Pooling层的训练~\ref{Cnn:9}
	\item 编程实战:卷积神经网络的实现~\ref{Cnn:10}
	\item 卷积层的实现~\ref{Cnn:11}
	\item Max Pooling层的实现~\ref{Cnn:12}
	\item 卷积神经网络的应用~\ref{Cnn:13}
\end{introduction}

在前面的文章中,我们介绍了全连接神经网络,以及它的训练和使用。我们用它来识别了手写数字,然而,这种结构的网络对于图像识别任务来说并不是很合适。本文将要介绍一种更适合图像、语音识别任务的神经网络结构------\textbf{卷积神经网络}(Convolutional Neural Network,
CNN)。说卷积神经网络是最重要的一种神经网络也不为过,它在最近几年大放异彩,几乎所有图像、语音识别领域的重要突破都是卷积神经网络取得的,比如谷歌的GoogleNet、微软的ResNet等,打败李世石的AlphaGo也用到了这种网络。本文将详细介绍\textbf{卷积神经网络}以及它的训练算法,以及动手实现一个简单的\textbf{卷积神经网络}。

\section{激活函数:Relu}\label{Cnn:1}

最近几年卷积神经网络中,激活函数往往不选择sigmoid或tanh函数,而是选择relu函数。Relu函数的定义是:
\[
	f(x)= max(0,x)
\]

Relu函数图像如图\ref{fig:Cnn1}所示:
\begin{figure}[!h]
	\centering
	\includegraphics[width=0.45\textwidth]{Cnn1.png}
	\caption{Relu函数}
	\label{fig:Cnn1}
\end{figure}


Relu函数作为激活函数,有下面几大优势:

\begin{itemize}
	\item
	      \textbf{速度快}
	      和sigmoid函数需要计算指数和倒数相比,relu函数其实就是一个max(0,x),计算代价小很多。
	\item
	      \textbf{减轻梯度消失问题}
	      回忆一下计算梯度的公式\(\nabla=\sigma'\delta x\)。其中,\(\sigma'\)是sigmoid函数的导数。在使用反向传播算法进行梯度计算时,每经过一层sigmoid神经元,梯度就要乘上一个\(\sigma'\)。从下图可以看出,\(\sigma'\)函数最大值是1/4。因此,乘一个\(\sigma'\)会导致梯度越来越小,这对于深层网络的训练是个很大的问题。而relu函数的导数是1,不会导致梯度变小。当然,激活函数仅仅是导致梯度减小的一个因素,但无论如何在这方面relu的表现强于sigmoid。使用relu激活函数可以让你训练更深的网络。
	\item
	      \textbf{稀疏性}
	      通过对大脑的研究发现,大脑在工作的时候只有大约5\%的神经元是激活的,而采用sigmoid激活函数的人工神经网络,其激活率大约是50\%。有论文声称人工神经网络在15\%-30\%的激活率时是比较理想的。因为relu函数在输入小于0时是完全不激活的,因此可以获得一个更低的激活率。
\end{itemize}

\begin{figure}[htbp]
	\centering
	\includegraphics[width=0.6\textwidth]{Cnn2.png}
	\caption{Relu函数导数}
	\label{fig:Cnn2}
\end{figure}



\section{全连接网络 VS 卷积网络}\label{Cnn:2}


全连接神经网络之所以不太适合图像识别任务,主要有以下几个方面的问题:

\begin{itemize}
	\item
	      \textbf{参数数量太多}
	      考虑一个输入1000*1000像素的图片(一百万像素,现在已经不能算大图了),输入层有1000*1000=100万节点。假设第一个隐藏层有100个节点(这个数量并不多),那么仅这一层就有(1000*1000+1)*100=1亿参数,这实在是太多了!我们看到图像只扩大一点,参数数量就会多很多,因此它的扩展性很差。
	\item
	      \textbf{没有利用像素之间的位置信息}
	      对于图像识别任务来说,每个像素和其周围像素的联系是比较紧密的,和离得很远的像素的联系可能就很小了。如果一个神经元和上一层所有神经元相连,那么就相当于对于一个像素来说,把图像的所有像素都等同看待,这不符合前面的假设。当我们完成每个连接权重的学习之后,最终可能会发现,有大量的权重,它们的值都是很小的(也就是这些连接其实无关紧要)。努力学习大量并不重要的权重,这样的学习必将是非常低效的。
	\item
	      \textbf{网络层数限制}
	      我们知道网络层数越多其表达能力越强,但是通过梯度下降方法训练深度全连接神经网络很困难,因为全连接神经网络的梯度很难传递超过3层。因此,我们不可能得到一个很深的全连接神经网络,也就限制了它的能力。
\end{itemize}

那么,卷积神经网络又是怎样解决这个问题的呢?主要有三个思路:

\begin{itemize}
	\item
	      \textbf{局部连接}
	      这个是最容易想到的,每个神经元不再和上一层的所有神经元相连,而只和一小部分神经元相连。这样就减少了很多参数。
	\item
	      \textbf{权值共享}
	      一组连接可以共享同一个权重,而不是每个连接有一个不同的权重,这样又减少了很多参数。
	\item
	      \textbf{下采样}
	      可以使用Pooling来减少每层的样本数,进一步减少参数数量,同时还可以提升模型的鲁棒性。
\end{itemize}

对于图像识别任务来说,卷积神经网络通过尽可能保留重要的参数,去掉大量不重要的参数,来达到更好的学习效果。

接下来,我们将详述卷积神经网络到底是何方神圣。


\section{卷积神经网络是啥}\label{Cnn:3}
首先,我们先获取一个感性认识,图\ref{fig:Cnn3}是一个卷积神经网络的示意图:

\begin{figure}[htbp]
	\centering
	\includegraphics[width=1\textwidth]{Cnn3.png}
	\caption{卷积神经网络}
	\label{fig:Cnn3}
\end{figure}

\subsection{网络架构}

如图\ref{fig:Cnn3}所示,一个卷积神经网络由若干\textbf{卷积层}、\textbf{Pooling层}、\textbf{全连接层}组成。你可以构建各种不同的卷积神经网络,它的常用架构模式为:
\begin{lstlisting}[numbers=none]
    INPUT -> [[CONV]*N -> POOL?]*M -> [FC]*K
\end{lstlisting}

也就是N个卷积层叠加,然后(可选)叠加一个Pooling层,重复这个结构M次,最后叠加K个全连接层。

对于图\ref{fig:Cnn3}展示的卷积神经网络:
\begin{lstlisting}[numbers=none]
    INPUT -> CONV -> POOL -> CONV -> POOL -> FC -> FC
\end{lstlisting}

按照上述模式可以表示为:
\begin{lstlisting}[numbers=none]
    INPUT -> [[CONV]*1 -> POOL]*2 -> [FC]*2
\end{lstlisting}

也就是:\texttt{N=1,\ M=2,\ K=2}。

\subsection{三维的层结构}

从图\ref{fig:Cnn3}我们可以发现\textbf{卷积神经网络}的层结构和\textbf{全连接神经网络}的层结构有很大不同。\textbf{全连接神经网络}每层的神经元是按照\textbf{一维}排列的,也就是排成一条线的样子;而\textbf{卷积神经网络}每层的神经元是按照\textbf{三维}排列的,也就是排成一个长方体的样子,有\textbf{宽度}、\textbf{高度}和\textbf{深度}。

对于图\ref{fig:Cnn3}展示的神经网络,我们看到输入层的宽度和高度对应于输入图像的宽度和高度,而它的深度为1。接着,第一个卷积层对这幅图像进行了卷积操作(后面我们会讲如何计算卷积),得到了三个Feature Map。这里的"3"可能是让很多初学者迷惑的地方,实际上,就是这个卷积层包含三个Filter,也就是三套参数,每个Filter都可以把原始输入图像卷积得到一个Feature Map,三个Filter就可以得到三个Feature Map。至于一个卷积层可以有多少个Filter,那是可以自由设定的。也就是说,卷积层的Filter个数也是一个\textbf{超参数}。我们可以把Feature Map可以看做是通过卷积变换提取到的图像特征,三个Filter就对原始图像提取出三组不同的特征,也就是得到了三个Feature Map,也称做三个\textbf{通道(channel)}。

继续观察图\ref{fig:Cnn3},在第一个卷积层之后,Pooling层对三个Feature Map做了\textbf{下采样}(后面我们会讲如何计算下采样),得到了三个更小的Feature Map。接着,是第二个\textbf{卷积层},它有5个Filter。每个Fitler都把前面\textbf{下采样}之后的\textbf{3个Feature Map}卷积\textbf{在一起,得到一个新的Feature Map。这样,5个Filter就得到了5个Feature Map。接着,是第二个Pooling,继续对5个Feature Map进行}下采样,得到了5个更小的Feature Map。

图\ref{fig:Cnn3}所示网络的最后两层是全连接层。第一个全连接层的每个神经元,和上一层5个Feature Map中的每个神经元相连,第二个全连接层(也就是输出层)的每个神经元,则和第一个全连接层的每个神经元相连,这样得到了整个网络的输出。

至此,我们对\textbf{卷积神经网络}有了最基本的感性认识。接下来,我们将介绍\textbf{卷积神经网络}中各种层的计算和训练。


\section{卷积神经网络输出值的计算}\label{Cnn:4}
\subsection{卷积层输出值的计算}\label{Cnn:5}

我们用一个简单的例子来讲述如何计算\textbf{卷积},然后,我们抽象出\textbf{卷积层}的一些重要概念和计算方法。

\begin{figure}[!h]
	\centering
	\includegraphics[width=0.8\textwidth]{Cnn4.png}
	\caption{Feature Map}
	\label{fig:Cnn4}
\end{figure}

假设有一个5*5的图像,使用一个3*3的filter进行卷积,想得到一个3*3的Feature Map,如图\ref{fig:Cnn4}所示。为了清楚的描述\textbf{卷积}计算过程,我们首先对图像的每个像素进行编号,用\(x_{i,j}\)表示图像的第\(i\)行第\(j\)列元素;对filter的每个权重进行编号,用\(w_{m,n}\)表示第\(m\)行第\(n\)列权重,用\(w_b\)表示filter的\textbf{偏置项};对Feature Map的每个元素进行编号,用\(a_{i,j}\)表示Feature Map的第\(i\)行第\(j\)列元素;用\(f\)表示\textbf{激活函数}(这个例子选择\textbf{relu函数}作为激活函数)。然后,使用下列公式计算卷积:

\begin{equation}
	\label{eq:Cnn1}
	a_{i,j}=f(\sum_{m=0}^{2}\sum_{n=0}^{2}w_{m,n}x_{i+m,j+n}+w_b)
\end{equation}


例如,对于Feature Map左上角元素\(a_{0,0}\)来说,其卷积计算方法为:
\begin{align*}
	a_{0,0} & =f(\sum_{m=0}^{2}\sum_{n=0}^{2}w_{m,n}x_{m+0,n+0}+w_b)                           \\
	        & =relu(w_{0,0}x_{0,0}+w_{0,1}x_{0,1}+w_{0,2}x_{0,2}+w_{1,0}x_{1,0}+w_{1,1}x_{1,1} \\
	        & +w_{1,2}x_{1,2}+w_{2,0}x_{2,0}+w_{2,1}x_{2,1}+w_{2,2}x_{2,2}+w_b)                \\
	        & =relu(1+0+1+0+1+0+0+0+1+0)=relu(4)=4
\end{align*}

计算结果如图\ref{fig:Cnn5}所示:

\begin{figure}[!h]
	\centering
	\includegraphics[width=0.8\textwidth]{Cnn5.png}
	\caption{Feature Map}
	\label{fig:Cnn5}
\end{figure}

接下来,Feature Map的元素\(a_{0,1}\)的卷积计算方法为:
\begin{align*}
	a_{0,1} & =f(\sum_{m=0}^{2}\sum_{n=0}^{2}w_{m,n}x_{m+0,n+1}+w_b)                           \\
	        & =relu(w_{0,0}x_{0,1}+w_{0,1}x_{0,2}+w_{0,2}x_{0,3}+w_{1,0}x_{1,1}+w_{1,1}x_{1,2} \\
	        & +w_{1,2}x_{1,3}+w_{2,0}x_{2,1}+w_{2,1}x_{2,3}+w_{2,2}x_{2,3}+w_b)                \\
	        & =relu(1+0+0+0+1+0+0+0+1+0)                                                       \\
	        & =relu(3)=3
\end{align*}


计算结果如图\ref{fig:Cnn6}所示:

\begin{figure}[h]
	\centering
	\includegraphics[width=0.8\textwidth]{Cnn6.png}
	\caption{Feature Map}
	\label{fig:Cnn6}
\end{figure}

可以依次计算出Feature Map中所有元素的值。图\ref{fig:Cnn7}显示了整个Feature Map的计算过程:

\begin{figure}[h]
	\centering
	\includegraphics[width=0.6\textwidth]{Cnn7.jpg}
	\caption{Feature Map}
	\label{fig:Cnn7}
\end{figure}

上面的计算过程中,步幅(stride)为1。步幅可以设为大于1的数。例如,当步幅为2时,Feature Map计算如图\ref{fig:Cnn8}。


\begin{figure}[!h]
	\centering
	\subfigure{
		\begin{minipage}[t]{0.7\linewidth}
			\centering
			\includegraphics[width=4in]{Cnn8.png}
			%\caption{fig1}
		\end{minipage}%
	}%

	\subfigure{
		\begin{minipage}[t]{0.7\linewidth}
			\centering
			\includegraphics[width=4in]{Cnn9.png}
			%\caption{fig2}
		\end{minipage}%
	}%

	\subfigure{
		\begin{minipage}[t]{0.7\linewidth}
			\centering
			\includegraphics[width=4in]{Cnn10.png}
			%\caption{fig2}
		\end{minipage}%
	}%

	\subfigure{
		\begin{minipage}[t]{0.7\linewidth}
			\centering
			\includegraphics[width=4in]{Cnn11.png}
			%\caption{fig2}
		\end{minipage}%
	}%
	\centering
	\caption{Feature Map计算过程}
	\label{fig:Cnn8}
\end{figure}



我们注意到,当\textbf{步幅}设置为2的时候,Feature Map就变成2*2了。这说明图像大小、步幅和卷积后的Feature Map大小是有关系的。事实上,它们满足下面的关系:
\begin{align}
	W_2 & = (W_1 - F + 2P)/S + 1\label{eq:Cnn2} \\
	H_2 & = (H_1 - F + 2P)/S + 1\label{eq:Cnn3}
\end{align}


在上面两个公式中,\(W_2\)是卷积后Feature Map的宽度;\(W_1\)是卷积前图像的宽度;\(F\)是filter的宽度;\(P\)是\textbf{Zero Padding}数量,\textbf{Zero Padding}是指在原始图像周围补几圈0,如果\(P\)的值是1,那么就补1圈0;\(S\)是\textbf{步幅};\(H_2\)是卷积后Feature Map的高度;\(H_1\)是卷积前图像的宽度。公式\ref{eq:Cnn2}和\ref{eq:Cnn3}本质上是一样的。

以前面的例子来说,图像宽度\(W_1=5\),filter宽度\(F=3\),\textbf{Zero Padding} \(P=0\),\textbf{步幅}\(S=2\),则
\begin{align*}
	W_2 = (W_1 - F + 2P)/S + 1= (5 - 3 + 0)/2 + 1=2
\end{align*}
说明Feature Map宽度是2。同样,我们也可以计算出Feature Map高度也是2。

前面我们已经讲了深度为1的卷积层的计算方法,如果深度大于1怎么计算呢?其实也是类似的。如果卷积前的图像深度为D,那么相应的filter的深度也必须为D。我们扩展一下公式\ref{eq:Cnn1},得到了深度大于1的卷积计算公式:
\begin{equation}
	\label{eq:Cnn4}
	a_{i,j}=f(\sum_{d=0}^{D-1}\sum_{m=0}^{F-1}\sum_{n=0}^{F-1}w_{d,m,n}x_{d,i+m,j+n}+w_b)
\end{equation}


在公式\ref{eq:Cnn4}中,D是深度;F是filter的大小(宽度或高度,两者相同);\(w_{d,m,n}\)表示filter的第\(d\)层第\(m\)行第\(n\)列权重;\(a_{d,i,j}\)表示图像的第\(d\)层第\(i\)行第\(j\)列像素;其它的符号含义和公式\ref{eq:Cnn1}是相同的,不再赘述。

我们前面还曾提到,每个卷积层可以有多个filter。每个filter和原始图像进行卷积后,都可以得到一个Feature Map。因此,卷积后Feature Map的深度(个数)和卷积层的filter个数是相同的。

\begin{figure}[!h]
	\centering
	\includegraphics[width=0.9\textwidth]{Cnn12.jpg}
	\caption{卷积过程}
	\label{fig:Cnn12}
\end{figure}

图\ref{fig:Cnn12}显示了包含两个filter的卷积层的计算。我们可以看到7*7*3输入,经过两个3*3*3 filter的卷积(步幅为2),得到了3*3*2的输出。另外我们也会看到下图的\textbf{Zero padding}是1,也就是在输入元素的周围补了一圈0。\textbf{Zero padding}对于图像边缘部分的特征提取是很有帮助的。


以上就是卷积层的计算方法。这里面体现了\textbf{局部连接}和\textbf{权值共享}:每层神经元只和上一层部分神经元相连(卷积计算规则),且filter的权值对于上一层所有神经元都是一样的。对于包含两个3*3*3的fitler的卷积层来说,其参数数量仅有(3*3*3+1)*2=56个,且参数数量与上一层神经元个数无关。与\textbf{全连接神经网络}相比,其参数数量大大减少了。

\textbf{用卷积公式来表达卷积层计算}

不想了解太多数学细节的读者可以跳过这一节,不影响对全文的理解。

公式\ref{eq:Cnn4}的表达很是繁冗,最好能简化一下。就像利用矩阵可以简化表达\textbf{全连接神经网络}的计算一样,我们利用\textbf{卷积公式}可以简化\textbf{卷积神经网络}的表达。

下面我们介绍\textbf{二维卷积公式}。

设矩阵\(A\),\(B\),其行、列数分别为\(m_a\)、\(n_a\)、\(m_b\)、\(n_b\),则\textbf{二维卷积公式}如下:
\begin{align*}
	C_{s,t} & =\sum_0^{m_a-1}\sum_0^{n_a-1} A_{m,n}B_{s-m,t-n}
\end{align*}

且\(s\),\(t\)满足条件$0 \leqslant {s} < {m_a+m_b-1}, 0 \leqslant {t} < {n_a+n_b-1}$。

我们可以把上式写成
\begin{equation}
	\label{eq:Cnn5}
	C = A * B
\end{equation}

如果我们按照公式\ref{eq:Cnn5}来计算卷积,我们可以发现矩阵A实际上是filter,而矩阵B是待卷积的输入,位置关系也有所不同如图\ref{fig:Cnn13}:
\begin{figure}[htbp]
	\centering
	\includegraphics[width=0.7\textwidth]{Cnn13.png}
	\caption{位置关系图}
	\label{fig:Cnn13}
\end{figure}


从上图可以看到,A左上角的值\(a_{0,0}\)与B对应区块中右下角的值\(b_{1,1}\)相乘,而不是与左上角的\(b_{0,0}\)相乘。因此,\textbf{数学}中的卷积和\textbf{卷积神经网络}中的『卷积』还是有区别的,为了避免混淆,我们把\textbf{卷积神经网络}中的『卷积』操作叫做\textbf{互相关(cross-correlation)}操作。

\textbf{卷积}和\textbf{互相关}操作是可以转化的。首先,我们把矩阵A翻转180度,然后再交换A和B的位置(即把B放在左边而把A放在右边。卷积满足交换率,这个操作不会导致结果变化),那么\textbf{卷积}就变成了\textbf{互相关}。

如果我们不去考虑两者这么一点点的区别,我们可以把公式\ref{eq:Cnn5}代入到公式\ref{eq:Cnn4}:
\begin{equation}
	\label{eq:Cnn6}
	A=f(\sum_{d=0}^{D-1}X_d*W_d+w_b)
\end{equation}
其中,\(A\)是卷积层输出的feature map。同公式\ref{eq:Cnn4}相比,公式\ref{eq:Cnn6}就简单多了。然而,这种简洁写法只适合步长为1的情况。


\subsection{Pooling层输出值的计算}\label{Cnn:6}

Pooling层主要的作用是\textbf{下采样},通过去掉Feature Map中不重要的样本,进一步减少参数数量。Pooling的方法很多,最常用的是\textbf{Max Pooling}。\textbf{Max Pooling}实际上就是在n*n的样本中取最大值,作为采样后的样本值。图\ref{fig:Cnn14}是2*2 max pooling。

\begin{figure}[htbp]
	\centering
	\includegraphics[width=0.7\textwidth]{Cnn14.png}
	\caption{max pooling}
	\label{fig:Cnn14}
\end{figure}

除了\textbf{Max Pooing}之外,常用的还有\textbf{Mean Pooling}---取各样本的平均值。

对于深度为D的Feature Map,各层独立做Pooling,因此Pooling后的深度仍然为D。


\subsection{全连接层}

全连接层输出值的计算和第\ref{chap:Bp}章\textbf{神经网络和反向传播算法}讲过的\textbf{全连接神经网络}是一样的,这里就不再赘述了。

\section{卷积神经网络的训练}\label{Cnn:7}

和\textbf{全连接神经网络}相比,\textbf{卷积神经网络}的训练要复杂一些。但训练的原理是一样的:利用链式求导计算损失函数对每个权重的偏导数(梯度),然后根据梯度下降公式更新权重。训练算法依然是反向传播算法。

我们先回忆一下第\ref{chap:Bp}章\textbf{神经网络和反向传播算法}介绍的反向传播算法,整个算法分为三个步骤:

\begin{enumerate}
	\item
	      前向计算每个神经元的\textbf{输出值}\(a_j\)(\(j\)表示网络的第\(j\)个神经元,以下同);
	\item
	      反向计算每个神经元的\textbf{误差项}\(\delta_j\),\(\delta_j\)在有的文献中也叫做\textbf{敏感度}(sensitivity)。它实际上是网络的损失函数\(E_d\)对神经元\textbf{加权输入}\(net_j\)的偏导数,即\(\delta_j=\frac{\partial{E_d}}{\partial{net_j}}\);
	\item
	      计算每个神经元连接权重\(w_{ji}\)的\textbf{梯度}(\(w_{ji}\)表示从神经元\(i\)连接到神经元\(j\)的权重),公式为\(\frac{\partial{E_d}}{\partial{w_{ji}}}=a_i\delta_j\),其中,\(a_i\)表示神经元\(i\)的输出。
\end{enumerate}

最后,根据梯度下降法则更新每个权重即可。

对于卷积神经网络,由于涉及到\textbf{局部连接}、\textbf{下采样}的等操作,影响到了第二步\textbf{误差项}\(\delta\)的具体计算方法,而\textbf{权值共享}影响了第三步\textbf{权重}\(w\)的\textbf{梯度}的计算方法。接下来,我们分别介绍卷积层和Pooling层的训练算法。

\subsection{卷积层的训练}\label{Cnn:8}

对于卷积层,我们先来看看上面的第二步,即如何将\textbf{误差项}\(\delta\)传递到上一层;然后再来看看第三步,即如何计算filter每个权值\(w\)的\textbf{梯度}。

\textbf{卷积层误差项的传递}

\textbf{最简单情况下误差项的传递}

我们先来考虑步长为1、输入的深度为1、filter个数为1的最简单的情况。

\begin{figure}[!h]
	\centering
	\includegraphics[width=0.7\textwidth]{Cnn15.png}
	\caption{feature map}
	\label{fig:Cnn15}
\end{figure}

假设输入的大小为3*3,filter大小为2*2,按步长为1卷积,我们将得到2*2的\textbf{feature map}。在图\ref{fig:Cnn15}中,为了描述方便,我们为每个元素都进行了编号。用\(\delta^{l-1}_{i,j}\)表示第\(l-1\)层第\(j\)行第\(j\)列的\textbf{误差项};用\(w_{m,n}\)表示filter第\(m\)行第\(n\)列权重,用\(w_b\)表示filter的\textbf{偏置项};用\(a^{l-1}_{i,j}\)表示第\(l-1\)层第\(i\)行第\(j\)列神经元的\textbf{输出};用\(net^{l-1}_{i,j}\)表示第\(l-1\)行神经元的\textbf{加权输入};用\(\delta^l_{i,j}\)表示第\(l\)层第\(j\)行第\(j\)列的\textbf{误差项};用\(f^{l-1}\)表示第\(l-1\)层的\textbf{激活函数}。它们之间的关系如下:
\begin{align*}
	net^l         & =conv(W^l, a^{l-1})+w_b   \\
	a^{l-1}_{i,j} & =f^{l-1}(net^{l-1}_{i,j})
\end{align*}
上式中,\(net^l\)、\(W^l\)、\(a^{l-1}\)都是数组,\(W^l\)是由\(w_{m,n}\)组成的数组,\(conv\)表示卷积操作。

在这里,我们假设第\(l\)中的每个\(\delta^l\)值都已经算好,我们要做的是计算第\(l-1\)层每个神经元的\textbf{误差项}\(\delta^{l-1}\)。

根据链式求导法则:
\begin{align*}
	\delta^{l-1}_{i,j}=\frac{\partial{E_d}}{\partial{net^{l-1}_{i,j}}} =\frac{\partial{E_d}}{\partial{a^{l-1}_{i,j}}}\frac{\partial{a^{l-1}_{i,j}}}{\partial{net^{l-1}_{i,j}}}
\end{align*}

我们先求第一项\(\frac{\partial{E_d}}{\partial{a^{l-1}_{i,j}}}\)。我们先来看几个特例,然后从中总结出一般性的规律。

\begin{example}
	计算\(\frac{\partial{E_d}}{\partial{a^{l-1}_{1,1}}}\),\(a^{l-1}_{1,1}\)仅与\(net^l_{1,1}\)的计算有关:
	\begin{equation*}
		net^j_{1,1}=w_{1,1}a^{l-1}_{1,1}+w_{1,2}a^{l-1}_{1,2}+w_{2,1}a^{l-1}_{2,1}+w_{2,2}a^{l-1}_{2,2}+w_b
	\end{equation*}
\end{example}
因此:
\begin{align*}
	\frac{\partial{E_d}}{\partial{a^{l-1}_{1,1}}}=\frac{\partial{E_d}}{\partial{net^{l}_{1,1}}}\frac{\partial{net^{l}_{1,1}}}{\partial{a^{l-1}_{1,1}}}=\delta^l_{1,1}w_{1,1}
\end{align*}

\begin{example}
	计算\(\frac{\partial{E_d}}{\partial{a^{l-1}_{1,2}}}\),\(a^{l-1}_{1,2}\)与\(net^l_{1,1}\)和\(net^l_{1,2}\)的计算都有关:
	\begin{align*}
		net^j_{1,1} & =w_{1,1}a^{l-1}_{1,1}+w_{1,2}a^{l-1}_{1,2}+w_{2,1}a^{l-1}_{2,1}+w_{2,2}a^{l-1}_{2,2}+w_b \\
		net^j_{1,2} & =w_{1,1}a^{l-1}_{1,2}+w_{1,2}a^{l-1}_{1,3}+w_{2,1}a^{l-1}_{2,2}+w_{2,2}a^{l-1}_{2,3}+w_b
	\end{align*}
\end{example}

因此:
\begin{align*}
	\frac{\partial{E_d}}{\partial{a^{l-1}_{1,2}}}=\frac{\partial{E_d}}{\partial{net^{l}_{1,1}}}\frac{\partial{net^{l}_{1,1}}}{\partial{a^{l-1}_{1,2}}}+\frac{\partial{E_d}}{\partial{net^{l}_{1,2}}}\frac{\partial{net^{l}_{1,2}}}{\partial{a^{l-1}_{1,2}}}=\delta^l_{1,1}w_{1,2}+\delta^l_{1,2}w_{1,1}
\end{align*}

\begin{example}
	计算\(\frac{\partial{E_d}}{\partial{a^{l-1}_{2,2}}}\),\(a^{l-1}_{2,2}\)与\(net^l_{1,1}\)、\(net^l_{1,2}\)、\(net^l_{2,1}\)和\(net^l_{2,2}\)的计算都有关:
	\begin{align*}
		net^j_{1,1} & =w_{1,1}a^{l-1}_{1,1}+w_{1,2}a^{l-1}_{1,2}+w_{2,1}a^{l-1}_{2,1}+w_{2,2}a^{l-1}_{2,2}+w_b \\
		net^j_{1,2} & =w_{1,1}a^{l-1}_{1,2}+w_{1,2}a^{l-1}_{1,3}+w_{2,1}a^{l-1}_{2,2}+w_{2,2}a^{l-1}_{2,3}+w_b \\
		net^j_{2,1} & =w_{1,1}a^{l-1}_{2,1}+w_{1,2}a^{l-1}_{2,2}+w_{2,1}a^{l-1}_{3,1}+w_{2,2}a^{l-1}_{3,2}+w_b \\
		net^j_{2,2} & =w_{1,1}a^{l-1}_{2,2}+w_{1,2}a^{l-1}_{2,3}+w_{2,1}a^{l-1}_{3,2}+w_{2,2}a^{l-1}_{3,3}+w_b
	\end{align*}
\end{example}


因此:
\begin{align*}
	\frac{\partial{E_d}}{\partial{a^{l-1}_{2,2}}} & =\frac{\partial{E_d}}{\partial{net^{l}_{1,1}}}\frac{\partial{net^{l}_{1,1}}}{\partial{a^{l-1}_{2,2}}}+\frac{\partial{E_d}}{\partial{net^{l}_{1,2}}}\frac{\partial{net^{l}_{1,2}}}{\partial{a^{l-1}_{2,2}}}+\frac{\partial{E_d}}{\partial{net^{l}_{2,1}}}\frac{\partial{net^{l}_{2,1}}}{\partial{a^{l-1}_{2,2}}}+\frac{\partial{E_d}}{\partial{net^{l}_{2,2}}}\frac{\partial{net^{l}_{2,2}}}{\partial{a^{l-1}_{2,2}}} \\
	                                              & =\delta^l_{1,1}w_{2,2}+\delta^l_{1,2}w_{2,1}+\delta^l_{2,1}w_{1,2}+\delta^l_{2,2}w_{1,1}
\end{align*}

从上面三个例子,我们发挥一下想象力,不难发现,计算\(\frac{\partial{E_d}}{\partial{a^{l-1}}}\),相当于把第\(l\)层的sensitive map周围补一圈0,在与180度翻转后的filter进行\textbf{cross-correlation},就能得到想要结果,如图\ref{fig:Cnn16}所示。

\begin{figure}[!h]
	\centering
	\includegraphics[width=0.7\textwidth]{Cnn16.png}
	\caption{cross correlation}
	\label{fig:Cnn16}
\end{figure}

因为\textbf{卷积}相当于将filter旋转180度的\textbf{cross-correlation},因此上图的计算可以用卷积公式完美的表达:
\[
	\frac{\partial{E_d}}{\partial{a_l}}=\delta^l*W^l
\]

上式中的\(W^l\)表示第\(l\)层的filter的权重数组。也可以把上式的卷积展开,写成求和的形式:
\[
	\frac{\partial{E_d}}{\partial{a^l_{i,j}}}=\sum_m\sum_n{w^l_{m,n}\delta^l_{i+m,j+n}}
\]

现在,我们再求第二项\(\frac{\partial{a^{l-1}_{i,j}}}{\partial{net^{l-1}_{i,j}}}\)。因为
\(
a^{l-1}_{i,j}=f(net^{l-1}_{i,j})
\)
所以这一项极其简单,仅求激活函数\(f\)的导数就行了。
\[
	\frac{\partial{a^{l-1}_{i,j}}}{\partial{net^{l-1}_{i,j}}}=f'(net^{l-1}_{i,j})
\]

将第一项和第二项组合起来,我们得到最终的公式:
\begin{equation}
	\label{eq:Cnn7}
	\delta^{l-1}_{i,j}=\frac{\partial{E_d}}{\partial{net^{l-1}_{i,j}}}=\frac{\partial{E_d}}{\partial{a^{l-1}_{i,j}}}\frac{\partial{a^{l-1}_{i,j}}}{\partial{net^{l-1}_{i,j}}}=\sum_m\sum_n{w^l_{m,n}\delta^l_{i+m,j+n}}f'(net^{l-1}_{i,j})
\end{equation}

也可以将公式\ref{eq:Cnn7}写成卷积的形式:
\begin{equation}
	\label{eq:Cnn8}
	\delta^{l-1}=\delta^l*W^l\circ f'(net^{l-1})\
\end{equation}
其中,符号\(\circ\)表示\textbf{element-wise product},即将矩阵中每个对应元素相乘。注意公式\ref{eq:Cnn8}中的\(\delta^{l-1}\)、\(\delta^l\)、\(net^{l-1}\)都是\textbf{矩阵}。

以上就是步长为1、输入的深度为1、filter个数为1的最简单的情况,卷积层误差项传递的算法。下面我们来推导一下步长为S的情况。

\textbf{卷积步长为S时的误差传递}

我们先来看看步长为S与步长为1的差别。

\begin{figure}[!h]
	\centering
	\includegraphics[width=0.75\textwidth]{Cnn17.png}
	\caption{卷积步长的差别}
	\label{fig:Cnn17}
\end{figure}

如图\ref{fig:Cnn17},上面是步长为1时的卷积结果,下面是步长为2时的卷积结果。我们可以看出,因为步长为2,得到的feature map跳过了步长为1时相应的部分。因此,当我们反向计算\textbf{误差项}时,我们可以对步长为S的sensitivity map相应的位置进行补0,将其『还原』成步长为1时的sensitivity map,再用公式\ref{eq:Cnn8}进行求解。

\textbf{输入层深度为D时的误差传递}

当输入深度为D时,filter的深度也必须为D,\(l-1\)层的\(d_i\)通道只与filter的\(d_i\)通道的权重进行计算。因此,反向计算\textbf{误差项}时,我们可以使用\textbf{式8},用filter的第\(d_i\)通道权重对第\(l\)层sensitivity map进行卷积,得到第\(l-1\)层\(d_i\)通道的sensitivity map。如图\ref{fig:Cnn18}所示:

\begin{figure}[!h]
	\centering
	\includegraphics[width=0.8\textwidth]{Cnn18.png}
	\caption{sensitivity map 卷积}
	\label{fig:Cnn18}
\end{figure}

\textbf{filter数量为N时的误差传递}

filter数量为N时,输出层的深度也为N,第\(i\)个filter卷积产生输出层的第\(i\)个feature map。由于第\(l-1\)层\textbf{每个加权输入}\(net^{l-1}_{d, i,j}\)都同时影响了第\(l\)层所有feature map的输出值,因此,反向计算\textbf{误差项}时,需要使用全导数公式。也就是,我们先使用第\(d\)个filter对第\(l\)层相应的第\(d\)个sensitivity map进行卷积,得到一组N个\(l-1\)层的偏sensitivity map。依次用每个filter做这种卷积,就得到D组偏sensitivity map。最后在各组之间将N个偏sensitivity map
\textbf{按元素相加},得到最终的N个\(l-1\)层的sensitivity map:

\begin{equation}
	\label{eq:Cnn9}
	\delta^{l-1}=\sum_{d=0}^D\delta_d^l*W_d^l\circ f'(net^{l-1})
\end{equation}

以上就是卷积层误差项传递的算法,如果读者还有所困惑,可以参考后面的代码实现来理解。

\textbf{卷积层filter权重梯度的计算}

我们要在得到第\(l\)层sensitivity map的情况下,计算filter的权重的梯度,由于卷积层是\textbf{权重共享}的,因此梯度的计算稍有不同。

\begin{figure}[!h]
	\centering
	\includegraphics[width=0.8\textwidth]{Cnn19.png}
	\caption{梯度的计算}
	\label{fig:Cnn19}
\end{figure}

如图\ref{fig:Cnn19}所示,\(a^l_{i,j}\)是第\(l-1\)层的输出,\(w_{i,j}\)是第\(l\)层filter的权重,\(\delta^l_{i,j}\)是第\(l\)层的sensitivity map。我们的任务是计算\(w_{i,j}\)的梯度,即\(\frac{\partial{E_d}}{\partial{w_{i,j}}}\)。

为了计算偏导数,我们需要考察权重\(w_{i,j}\)对\(E_d\)的影响。权重项\(w_{i,j}\)通过影响\(net^l_{i,j}\)的值,进而影响\(E_d\)。我们仍然通过几个具体的例子来看权重项\(w_{i,j}\)对\(net^l_{i,j}\)的影响,然后再从中总结出规律。

\begin{example}
	计算\(\frac{\partial{E_d}}{\partial{w_{1,1}}}\):
\end{example}

\begin{align*}
	net^j_{1,1}=w_{1,1}a^{l-1}_{1,1}+w_{1,2}a^{l-1}_{1,2}+w_{2,1}a^{l-1}_{2,1}+w_{2,2}a^{l-1}_{2,2}+w_b \\
	net^j_{1,2}=w_{1,1}a^{l-1}_{1,2}+w_{1,2}a^{l-1}_{1,3}+w_{2,1}a^{l-1}_{2,2}+w_{2,2}a^{l-1}_{2,3}+w_b \\
	net^j_{2,1}=w_{1,1}a^{l-1}_{2,1}+w_{1,2}a^{l-1}_{2,2}+w_{2,1}a^{l-1}_{3,1}+w_{2,2}a^{l-1}_{3,2}+w_b \\
	net^j_{2,2}=w_{1,1}a^{l-1}_{2,2}+w_{1,2}a^{l-1}_{2,3}+w_{2,1}a^{l-1}_{3,2}+w_{2,2}a^{l-1}_{3,3}+w_b
\end{align*}
从上面的公式看出,由于\textbf{权值共享},权值\(w_{1,1}\)对所有的\(net^l_{i,j}\)都有影响。\(E_d\)是\(net^l_{1,1}\)、\(net^l_{1,2}\)、\(net^l_{2,1}\)...的函数,而\(net^l_{1,1}\)、\(net^l_{1,2}\)、\(net^l_{2,1}\)...又是\(w_{1,1}\)的函数,根据\textbf{全导数}公式,计算\(\frac{\partial{E_d}}{\partial{w_{1,1}}}\)就是要把每个偏导数都加起来:
\begin{align*}
	\frac{\partial{E_d}}{\partial{w_{1,1}}} & =\frac{\partial{E_d}}{\partial{net^{l}_{1,1}}}\frac{\partial{net^{l}_{1,1}}}{\partial{w_{1,1}}}+\frac{\partial{E_d}}{\partial{net^{l}_{1,2}}}\frac{\partial{net^{l}_{1,2}}}{\partial{w_{1,1}}}+\frac{\partial{E_d}}{\partial{net^{l}_{2,1}}}\frac{\partial{net^{l}_{2,1}}}{\partial{w_{1,1}}}+\frac{\partial{E_d}}{\partial{net^{l}_{2,2}}}\frac{\partial{net^{l}_{2,2}}}{\partial{w_{1,1}}} \\
	                                        & =\delta^l_{1,1}a^{l-1}_{1,1}+\delta^l_{1,2}a^{l-1}_{1,2}+\delta^l_{2,1}a^{l-1}_{2,1}+\delta^l_{2,2}a^{l-1}_{2,2}
\end{align*}

\begin{example}
	计算\(\frac{\partial{E_d}}{\partial{w_{1,2}}}\):
\end{example}

通过查看\(w_{1,2}\)与\(net^l_{i,j}\)的关系,我们很容易得到:
\[
	\frac{\partial{E_d}}{\partial{w_{1,2}}}=\delta^l_{1,1}a^{l-1}_{1,2}+\delta^l_{1,2}a^{l-1}_{1,3}+\delta^l_{2,1}a^{l-1}_{2,2}+\delta^l_{2,2}a^{l-1}_{2,3}
\]

实际上,每个\textbf{权重项}都是类似的,我们不一一举例了。现在,是我们再次发挥想象力的时候,我们发现计算\(\frac{\partial{E_d}}{\partial{w_{i,j}}}\)规律是:
\[
	\frac{\partial{E_d}}{\partial{w_{i,j}}}=\sum_m\sum_n\delta_{m,n}a^{l-1}_{i+m,j+n}
\]

也就是用sensitivity map作为卷积核,在input上进行\textbf{cross-correlation},如图\ref{fig:Cnn20}所示。

\begin{figure}[h]
	\centering
	\includegraphics[width=0.7\textwidth]{Cnn20.png}
	\caption{cross correlation}
	\label{fig:Cnn20}
\end{figure}

最后,我们来看一看偏置项的梯度\(\frac{\partial{E_d}}{\partial{w_b}}\)。通过查看前面的公式,我们很容易发现:
\begin{align*}
	\frac{\partial{E_d}}{\partial{w_b}} & =\frac{\partial{E_d}}{\partial{net^{l}_{1,1}}}\frac{\partial{net^{l}_{1,1}}}{\partial{w_b}}+\frac{\partial{E_d}}{\partial{net^{l}_{1,2}}}\frac{\partial{net^{l}_{1,2}}}{\partial{w_b}}+\frac{\partial{E_d}}{\partial{net^{l}_{2,1}}}\frac{\partial{net^{l}_{2,1}}}{\partial{w_b}}+\frac{\partial{E_d}}{\partial{net^{l}_{2,2}}}\frac{\partial{net^{l}_{2,2}}}{\partial{w_b}} \\
	                                    & =\delta^l_{1,1}+\delta^l_{1,2}+\delta^l_{2,1}+\delta^l_{2,2}=\sum_i\sum_j\delta^l_{i,j}
\end{align*}


也就是\textbf{偏置项}的\textbf{梯度}就是sensitivity map所有\textbf{误差项}之和。

对于步长为S的卷积层,处理方法与传递\textbf{误差项}是一样的,首先将sensitivity map『还原』成步长为1时的sensitivity map,再用上面的方法进行计算。

获得了所有的\textbf{梯度}之后,就是根据\textbf{梯度下降算法}来更新每个权重。这在前面的文章中已经反复写过,这里就不再重复了。

至此,我们已经解决了卷积层的训练问题,接下来我们看一看Pooling层的训练。



\subsection{Pooling层的训练}\label{Cnn:9}

无论max pooling还是mean
pooling,都没有需要学习的参数。因此,在\textbf{卷积神经网络}的训练中,Pooling层需要做的仅仅是将\textbf{误差项}传递到上一层,而没有\textbf{梯度}的计算。

\textbf{Max Pooling误差项的传递}

\begin{figure}[!h]
	\centering
	\includegraphics[width=0.6\textwidth]{Cnn21.png}
	\caption{cross correlation}
	\label{fig:Cnn21}
\end{figure}

如图\ref{fig:Cnn21},假设第\(l-1\)层大小为4*4,pooling filter大小为2*2,步长为2,这样,max pooling 之后,第\(l\)层大小为2*2。假设第\(l\)层的\(\delta\)值都已经计算完毕,我们现在的任务是计算第\(l-1\)层的\(\delta\)值。


我们用\(net^{l-1}_{i,j}\)表示第\(l-1\)层的\textbf{加权输入};用\(net^l_{i,j}\)表示第\(l\)层的\textbf{加权输入}。我们先来考察一个具体的例子,然后再总结一般性的规律。对于max pooling:
\[
	net^l_{1,1}=max(net^{l-1}_{1,1},net^{l-1}_{1,2},net^{l-1}_{2,1},net^{l-1}_{2,2})
\]
也就是说,只有区块中最大的\(net^{l-1}_{i,j}\)才会对\(net^l_{i,j}\)的值产生影响。我们假设最大的值是\(net^{l-1}_{1,1}\),则上式相当于:
\[
	net^l_{1,1}=net^{l-1}_{1,1}
\]
那么,我们不难求得下面几个偏导数:
\begin{align*}
	\frac{\partial{net^l_{1,1}}}{\partial{net^{l-1}_{1,1}}}=1, \quad \frac{\partial{net^l_{1,1}}}{\partial{net^{l-1}_{1,2}}}=0 \\
	\frac{\partial{net^l_{1,1}}}{\partial{net^{l-1}_{2,1}}}=0, \quad \frac{\partial{net^l_{1,1}}}{\partial{net^{l-1}_{2,2}}}=0
\end{align*}

因此:
\begin{align*}
	\delta^{l-1}_{1,1} & =\frac{\partial{E_d}}{\partial{net^{l-1}_{1,1}}}=\frac{\partial{E_d}}{\partial{net^{l}_{1,1}}}\frac{\partial{net^{l}_{1,1}}}{\partial{net^{l-1}_{1,1}}}=\delta^{l}_{1,1}, \quad \delta^{l-1}_{1,2}=\frac{\partial{E_d}}{\partial{net^{l-1}_{1,2}}}=\frac{\partial{E_d}}{\partial{net^{l}_{1,1}}}\frac{\partial{net^{l}_{1,1}}}{\partial{net^{l-1}_{1,2}}}=0 \\
	\delta^{l-1}_{2,1} & =\frac{\partial{E_d}}{\partial{net^{l-1}_{2,1}}}=\frac{\partial{E_d}}{\partial{net^{l}_{1,1}}}\frac{\partial{net^{l}_{1,1}}}{\partial{net^{l-1}_{2,1}}}=0, \quad \delta^{l-1}_{1,1}=\frac{\partial{E_d}}{\partial{net^{l-1}_{2,2}}}=\frac{\partial{E_d}}{\partial{net^{l}_{1,1}}}\frac{\partial{net^{l}_{1,1}}}{\partial{net^{l-1}_{2,2}}}=0
\end{align*}

现在,我们发现了规律:对于max pooling,下一层的\textbf{误差项}的值会原封不动的传递到上一层对应区块中的最大值所对应的神经元,而其他神经元的\textbf{误差项}的值都是0。如图\ref{fig:Cnn22}所示(假设\(a^{l-1}_{1,1}\)、\(a^{l-1}_{1,4}\)、\(a^{l-1}_{4,1}\)、\(a^{l-1}_{4,4}\)为所在区块中的最大输出值)。

\begin{figure}[!h]
	\centering
	\includegraphics[width=0.6\textwidth]{Cnn22.png}
	\caption{max pooling}
	\label{fig:Cnn22}
\end{figure}

\textbf{Mean Pooling误差项的传递}

我们还是用前面屡试不爽的套路,先研究一个特殊的情形,再扩展为一般规律。
如图\ref{fig:Cnn23},我们先来考虑计算\(\delta^{l-1}_{1,1}\)。我们先来看看\(net^{l-1}_{1,1}\)如何影响\(net^l_{1,1}\)。
\[
	net^j_{1,1}=\frac{1}{4}(net^{l-1}_{1,1}+net^{l-1}_{1,2}+net^{l-1}_{2,1}+net^{l-1}_{2,2})
\]

\begin{figure}[!h]
	\centering
	\includegraphics[width=0.6\textwidth]{Cnn23.png}
	\caption{Mean Pooling}
	\label{fig:Cnn23}
\end{figure}
根据上式,我们一眼就能看出来:
\[
	\frac{\partial{net^l_{1,1}}}{\partial{net^{l-1}_{1,1}}}=\frac{1}{4}\\
	\frac{\partial{net^l_{1,1}}}{\partial{net^{l-1}_{1,2}}}=\frac{1}{4}\\
	\frac{\partial{net^l_{1,1}}}{\partial{net^{l-1}_{2,1}}}=\frac{1}{4}\\
	\frac{\partial{net^l_{1,1}}}{\partial{net^{l-1}_{2,2}}}=\frac{1}{4}\\
\]

所以,根据链式求导法则,我们不难算出:
\begin{align*}
	\delta^{l-1}_{1,1} & =\frac{\partial{E_d}}{\partial{net^{l-1}_{1,1}}}=\frac{\partial{E_d}}{\partial{net^{l}_{1,1}}}\frac{\partial{net^{l}_{1,1}}}{\partial{net^{l-1}_{1,1}}}=\frac{1}{4}\delta^{l}_{1,1}, \quad
	\delta^{l-1}_{1,2}=\frac{\partial{E_d}}{\partial{net^{l-1}_{1,2}}}=\frac{\partial{E_d}}{\partial{net^{l}_{1,1}}}\frac{\partial{net^{l}_{1,1}}}{\partial{net^{l-1}_{1,2}}}=\frac{1}{4}\delta^{l}_{1,1}           \\
	\delta^{l-1}_{2,1} & =\frac{\partial{E_d}}{\partial{net^{l-1}_{2,1}}}=\frac{\partial{E_d}}{\partial{net^{l}_{1,1}}}\frac{\partial{net^{l}_{1,1}}}{\partial{net^{l-1}_{2,1}}}=\frac{1}{4}\delta^{l}_{1,1}, \quad
	\delta^{l-1}_{2,2}=\frac{\partial{E_d}}{\partial{net^{l-1}_{2,2}}}=\frac{\partial{E_d}}{\partial{net^{l}_{1,1}}}\frac{\partial{net^{l}_{1,1}}}{\partial{net^{l-1}_{2,2}}}=\frac{1}{4}\delta^{l}_{1,1}
\end{align*}


现在,我们发现了规律:对于mean pooling,下一层的\textbf{误差项}的值会\textbf{平均分配}到上一层对应区块中的所有神经元。如图\ref{fig:Cnn24}所示。
\begin{figure}[!h]
	\centering
	\includegraphics[width=0.6\textwidth]{Cnn24.png}
	\caption{Mean Pooling}
	\label{fig:Cnn24}
\end{figure}

上面这个算法可以表达为高大上的\textbf{克罗内克积(Kronecker product)}的形式,有兴趣的读者可以研究一下。
\[
	\delta^{l-1} = \delta^l\otimes(\frac{1}{n^2})_{n\times n}
\]
其中,\(n\)是pooling层filter的大小,\(\delta^{l-1}\)、\( \delta^l\)都是矩阵。

至此,我们已经把\textbf{卷积层}、\textbf{Pooling层}的训练算法介绍完毕,加上上一篇文章讲的\textbf{全连接层}训练算法,您应该已经具备了编写\textbf{卷积神经网络}代码所需要的知识。为了加深对知识的理解,接下来,我们将展示如何实现一个简单的\textbf{卷积神经网络}。



\section{编程实战:卷积神经网络的实现}\label{Cnn:10}

\begin{note}
	完整代码请参考GitHub: \url{https://github.com/hanbt/learn_dl/blob/master/cnn.py}
	(python2.7)
\end{note}

现在,我们亲自动手实现一个卷积神经网络,以便巩固我们所学的知识。

首先,我们要改变一下代码的架构,『层』成为了我们最核心的组件。这是因为卷积神经网络有不同的层,而每种层的算法都在对应的类中实现。

这次,我们用到了在python中编写算法经常会用到的\textbf{numpy}包。为了使用\textbf{numpy},我们需要先将\textbf{numpy}导入:
\begin{lstlisting}
import numpy as np
\end{lstlisting}

\subsection{卷积层的实现}\label{Cnn:11}

\textbf{卷积层初始化}

我们用\textbf{ConvLayer}类来实现一个卷积层。下面的代码是初始化一个卷积层,可以在构造函数中设置卷积层的\textbf{超参数}。
\begin{lstlisting}
class ConvLayer(object):
    def __init__(self, input_width, input_height, 
                 channel_number, filter_width, 
                 filter_height, filter_number, 
                 zero_padding, stride, activator,
                 learning_rate):
        self.input_width = input_width
        self.input_height = input_height
        self.channel_number = channel_number
        self.filter_width = filter_width
        self.filter_height = filter_height
        self.filter_number = filter_number
        self.zero_padding = zero_padding
        self.stride = stride
        self.output_width = \
            ConvLayer.calculate_output_size(
            self.input_width, filter_width, zero_padding,
            stride)
        self.output_height = \
            ConvLayer.calculate_output_size(
            self.input_height, filter_height, zero_padding,
            stride)
        self.output_array = np.zeros((self.filter_number, 
            self.output_height, self.output_width))
        self.filters = []
        for i in range(filter_number):
            self.filters.append(Filter(filter_width, 
                filter_height, self.channel_number))
        self.activator = activator
        self.learning_rate = learning_rate
\end{lstlisting}

\textbf{calculate\_output\_size}函数用来确定卷积层输出的大小,其实现如下:
\begin{lstlisting}
    @staticmethod
    def calculate_output_size(input_size,
            filter_size, zero_padding, stride):
        return (input_size - filter_size + 
            2 * zero_padding) / stride + 1
\end{lstlisting}

\textbf{Filter}类保存了卷积层的\textbf{参数}以及\textbf{梯度},并且实现了用\textbf{梯度下降算法}来更新参数。
\begin{lstlisting}
class Filter(object):
    def __init__(self, width, height, depth):
        self.weights = np.random.uniform(-1e-4, 1e-4,
            (depth, height, width))
        self.bias = 0
        self.weights_grad = np.zeros(
            self.weights.shape)
        self.bias_grad = 0
    def __repr__(self):
        return 'filter weights:\n%s\nbias:\n%s' % (
            repr(self.weights), repr(self.bias))
    def get_weights(self):
        return self.weights
    def get_bias(self):
        return self.bias
    def update(self, learning_rate):
        self.weights -= learning_rate * self.weights_grad
        self.bias -= learning_rate * self.bias_grad
\end{lstlisting}

我们对参数的初始化采用了常用的策略,即:\textbf{权重}随机初始化为一个很小的值,而\textbf{偏置项}初始化为0。

\textbf{Activator}类实现了\textbf{激活函数},其中,\textbf{forward}方法实现了前向计算,而\textbf{backward}方法则是计算\textbf{导数}。比如,relu函数的实现如下:
\begin{lstlisting}
class ReluActivator(object):
    def forward(self, weighted_input):
        #return weighted_input
        return max(0, weighted_input)
    def backward(self, output):
        return 1 if output > 0 else 0
\end{lstlisting}

\textbf{卷积层前向计算的实现}

\textbf{ConvLayer}类的\textbf{forward}方法实现了卷积层的前向计算(即计算根据输入来计算卷积层的输出),下面是代码实现:
\begin{lstlisting}
    def forward(self, input_array):
        '''
        计算卷积层的输出
        输出结果保存在self.output_array
        '''
        self.input_array = input_array
        self.padded_input_array = padding(input_array,
            self.zero_padding)
        for f in range(self.filter_number):
            filter = self.filters[f]
            conv(self.padded_input_array, 
                filter.get_weights(), self.output_array[f],
                self.stride, filter.get_bias())
        element_wise_op(self.output_array, 
                        self.activator.forward)
\end{lstlisting}

上面的代码里面包含了几个工具函数。\textbf{element\_wise\_op}函数实现了对\textbf{numpy}数组进行\textbf{按元素}操作,并将返回值写回到数组中,代码如下:
\begin{lstlisting}
# 对numpy数组进行element wise操作
def element_wise_op(array, op):
    for i in np.nditer(array,
                       op_flags=['readwrite']):
        i[...] = op(i)
\end{lstlisting}

\textbf{conv}函数实现了2维和3维数组的\textbf{卷积},代码如下:
\begin{lstlisting}
def conv(input_array, 
         kernel_array,
         output_array, 
         stride, bias):
    '''
    计算卷积,自动适配输入为2D和3D的情况
    '''
    channel_number = input_array.ndim
    output_width = output_array.shape[1]
    output_height = output_array.shape[0]
    kernel_width = kernel_array.shape[-1]
    kernel_height = kernel_array.shape[-2]
    for i in range(output_height):
        for j in range(output_width):
            output_array[i][j] = (    
                get_patch(input_array, i, j, kernel_width, 
                    kernel_height, stride) * kernel_array
                ).sum() + bias
\end{lstlisting}

\textbf{padding}函数实现了zero padding操作:
\begin{lstlisting}
# 为数组增加Zero padding
def padding(input_array, zp):
    '''
    为数组增加Zero padding,自动适配输入为2D和3D的情况
    '''
    if zp == 0:
        return input_array
    else:
        if input_array.ndim == 3:
            input_width = input_array.shape[2]
            input_height = input_array.shape[1]
            input_depth = input_array.shape[0]
            padded_array = np.zeros((
                input_depth, 
                input_height + 2 * zp,
                input_width + 2 * zp))
            padded_array[:,
                zp : zp + input_height,
                zp : zp + input_width] = input_array
            return padded_array
        elif input_array.ndim == 2:
            input_width = input_array.shape[1]
            input_height = input_array.shape[0]
            padded_array = np.zeros((
                input_height + 2 * zp,
                input_width + 2 * zp))
            padded_array[zp : zp + input_height,
                zp : zp + input_width] = input_array
            return padded_array
\end{lstlisting}

\textbf{卷积层反向传播算法的实现}

现在,是介绍卷积层核心算法的时候了。我们知道反向传播算法需要完成几个任务:
\begin{enumerate}
	\item
	      将\textbf{误差项}传递到上一层。
	\item
	      计算每个\textbf{参数}的\textbf{梯度}。
	\item
	      更新\textbf{参数}。
\end{enumerate}

以下代码都是在\textbf{ConvLayer}类中实现。我们先来看看将\textbf{误差项}传递到上一层的代码实现。
\begin{lstlisting}
    def bp_sensitivity_map(self, sensitivity_array,
                           activator):
        '''
        计算传递到上一层的sensitivity map
        sensitivity_array: 本层的sensitivity map
        activator: 上一层的激活函数
        '''
        # 处理卷积步长,对原始sensitivity map进行扩展
        expanded_array = self.expand_sensitivity_map(
            sensitivity_array)
        # full卷积,对sensitivitiy map进行zero padding
        # 虽然原始输入的zero padding单元也会获得残差
        # 但这个残差不需要继续向上传递,因此就不计算了
        expanded_width = expanded_array.shape[2]
        zp = (self.input_width +  
              self.filter_width - 1 - expanded_width) / 2
        padded_array = padding(expanded_array, zp)
        # 初始化delta_array,用于保存传递到上一层的
        # sensitivity map
        self.delta_array = self.create_delta_array()
        # 对于具有多个filter的卷积层来说,最终传递到上一层的
        # sensitivity map相当于所有的filter的
        # sensitivity map之和
        for f in range(self.filter_number):
            filter = self.filters[f]
            # 将filter权重翻转180度
            flipped_weights = np.array(map(
                lambda i: np.rot90(i, 2), 
                filter.get_weights()))
            # 计算与一个filter对应的delta_array
            delta_array = self.create_delta_array()
            for d in range(delta_array.shape[0]):
                conv(padded_array[f], flipped_weights[d],
                    delta_array[d], 1, 0)
            self.delta_array += delta_array
        # 将计算结果与激活函数的偏导数做element-wise乘法操作
        derivative_array = np.array(self.input_array)
        element_wise_op(derivative_array, 
                        activator.backward)
        self.delta_array *= derivative_array
\end{lstlisting}

\textbf{expand\_sensitivity\_map}方法就是将步长为S的sensitivity
map『还原』为步长为1的sensitivity map,代码如下:
\begin{lstlisting}
    def expand_sensitivity_map(self, sensitivity_array):
        depth = sensitivity_array.shape[0]
        # 确定扩展后sensitivity map的大小
        # 计算stride为1时sensitivity map的大小
        expanded_width = (self.input_width - 
            self.filter_width + 2 * self.zero_padding + 1)
        expanded_height = (self.input_height - 
            self.filter_height + 2 * self.zero_padding + 1)
        # 构建新的sensitivity_map
        expand_array = np.zeros((depth, expanded_height, 
                                 expanded_width))
        # 从原始sensitivity map拷贝误差值
        for i in range(self.output_height):
            for j in range(self.output_width):
                i_pos = i * self.stride
                j_pos = j * self.stride
                expand_array[:,i_pos,j_pos] = \
                    sensitivity_array[:,i,j]
        return expand_array
\end{lstlisting}

\textbf{create\_delta\_array}是创建用来保存传递到上一层的sensitivity
map的数组。
\begin{lstlisting}
    def create_delta_array(self):
        return np.zeros((self.channel_number,
            self.input_height, self.input_width))
\end{lstlisting}

接下来,是计算梯度的代码。
\begin{lstlisting}
    def bp_gradient(self, sensitivity_array):
        # 处理卷积步长,对原始sensitivity map进行扩展
        expanded_array = self.expand_sensitivity_map(
            sensitivity_array)
        for f in range(self.filter_number):
            # 计算每个权重的梯度
            filter = self.filters[f]
            for d in range(filter.weights.shape[0]):
                conv(self.padded_input_array[d], 
                     expanded_array[f],
                     filter.weights_grad[d], 1, 0)
            # 计算偏置项的梯度
            filter.bias_grad = expanded_array[f].sum()
\end{lstlisting}

最后,是按照\textbf{梯度下降算法}更新参数的代码,这部分非常简单。
\begin{lstlisting}
    def update(self):
        '''
        按照梯度下降,更新权重
        '''
        for filter in self.filters:
            filter.update(self.learning_rate)
\end{lstlisting}

\textbf{卷积层的梯度检查}

为了验证我们的公式推导和代码实现的正确性,我们必须要对卷积层进行梯度检查。下面是代吗实现:
\begin{lstlisting}
def init_test():
    a = np.array(
        [[[0,1,1,0,2],
          [2,2,2,2,1],
          [1,0,0,2,0],
          [0,1,1,0,0],
          [1,2,0,0,2]],
         [[1,0,2,2,0],
          [0,0,0,2,0],
          [1,2,1,2,1],
          [1,0,0,0,0],
          [1,2,1,1,1]],
         [[2,1,2,0,0],
          [1,0,0,1,0],
          [0,2,1,0,1],
          [0,1,2,2,2],
          [2,1,0,0,1]]])
    b = np.array(
        [[[0,1,1],
          [2,2,2],
          [1,0,0]],
         [[1,0,2],
          [0,0,0],
          [1,2,1]]])
    cl = ConvLayer(5,5,3,3,3,2,1,2,IdentityActivator(),0.001)
    cl.filters[0].weights = np.array(
        [[[-1,1,0],
          [0,1,0],
          [0,1,1]],
         [[-1,-1,0],
          [0,0,0],
          [0,-1,0]],
         [[0,0,-1],
          [0,1,0],
          [1,-1,-1]]], dtype=np.float64)
    cl.filters[0].bias=1
    cl.filters[1].weights = np.array(
        [[[1,1,-1],
          [-1,-1,1],
          [0,-1,1]],
         [[0,1,0],
         [-1,0,-1],
          [-1,1,0]],
         [[-1,0,0],
          [-1,0,1],
          [-1,0,0]]], dtype=np.float64)
    return a, b, cl
def gradient_check():
    '''
    梯度检查
    '''
    # 设计一个误差函数,取所有节点输出项之和
    error_function = lambda o: o.sum()
    # 计算forward值
    a, b, cl = init_test()
    cl.forward(a)
    # 求取sensitivity map,是一个全1数组
    sensitivity_array = np.ones(cl.output_array.shape,
                                dtype=np.float64)
    # 计算梯度
    cl.backward(a, sensitivity_array,
                  IdentityActivator())
    # 检查梯度
    epsilon = 10e-4
    for d in range(cl.filters[0].weights_grad.shape[0]):
        for i in range(cl.filters[0].weights_grad.shape[1]):
            for j in range(cl.filters[0].weights_grad.shape[2]):
                cl.filters[0].weights[d,i,j] += epsilon
                cl.forward(a)
                err1 = error_function(cl.output_array)
                cl.filters[0].weights[d,i,j] -= 2*epsilon
                cl.forward(a)
                err2 = error_function(cl.output_array)
                expect_grad = (err1 - err2) / (2 * epsilon)
                cl.filters[0].weights[d,i,j] += epsilon
                print 'weights(%d,%d,%d): expected - actural %f - %f' % (
                    d, i, j, expect_grad, cl.filters[0].weights_grad[d,i,j])   
\end{lstlisting}

上面代码值得思考的地方在于,传递给卷积层的sensitivity
map是全1数组,留给读者自己推导一下为什么是这样(提示:激活函数选择了identity函数:{}\(f(x)=x\))。读者如果还有困惑,请写在文章评论中,我会回复。

运行上面梯度检查的代码,我们得到的输出如下,期望的梯度和实际计算出的梯度一致,这证明我们的算法推导和代码实现确实是正确的。

\includegraphics[width=0.7\textwidth]{Cnn25.png}

以上就是卷积层的实现。

\subsection{Max Pooling层的实现}\label{Cnn:12}

max pooling层的实现相对简单,我们直接贴出全部代码如下:
\begin{lstlisting}
class MaxPoolingLayer(object):
    def __init__(self, input_width, input_height, 
                 channel_number, filter_width, 
                 filter_height, stride):
        self.input_width = input_width
        self.input_height = input_height
        self.channel_number = channel_number
        self.filter_width = filter_width
        self.filter_height = filter_height
        self.stride = stride
        self.output_width = (input_width - 
            filter_width) / self.stride + 1
        self.output_height = (input_height -
            filter_height) / self.stride + 1
        self.output_array = np.zeros((self.channel_number,
            self.output_height, self.output_width))
    def forward(self, input_array):
        for d in range(self.channel_number):
            for i in range(self.output_height):
                for j in range(self.output_width):
                    self.output_array[d,i,j] = (    
                        get_patch(input_array[d], i, j,
                            self.filter_width, 
                            self.filter_height, 
                            self.stride).max())
    def backward(self, input_array, sensitivity_array):
        self.delta_array = np.zeros(input_array.shape)
        for d in range(self.channel_number):
            for i in range(self.output_height):
                for j in range(self.output_width):
                    patch_array = get_patch(
                        input_array[d], i, j,
                        self.filter_width, 
                        self.filter_height, 
                        self.stride)
                    k, l = get_max_index(patch_array)
                    self.delta_array[d, 
                        i * self.stride + k, 
                        j * self.stride + l] = \
                        sensitivity_array[d,i,j]
\end{lstlisting}

全连接层的实现和上一篇文章类似,在此就不再赘述了。至此,你已经拥有了实现了一个简单的\textbf{卷积神经网络}所需要的基本组件。对于\textbf{卷积神经网络},现在有很多优秀的开源实现,因此我们并不需要真的自己去实现一个。贴出这些代码的目的是为了让我们更好的了解\textbf{卷积神经网络}的基本原理。



\section{卷积神经网络的应用}\label{Cnn:13}

\textbf{MNIST手写数字识别}

\emph{LeNet-5}是实现手写数字识别的\textbf{卷积神经网络},在MNIST测试集上,它取得了0.8\%的错误率。\emph{LeNet-5}的结构如图\ref{fig:Cnn26}:

\begin{figure}[!h]
	\centering
	\includegraphics[width=1\textwidth]{Cnn26.png}
	\caption{LeNet-5}
	\label{fig:Cnn26}
\end{figure}

关于\emph{LeNet-5}的详细介绍,网上的资料很多,因此就不再重复了。感兴趣的读者可以尝试用我们自己实现的卷积神经网络代码去构造并训练\emph{LeNet-5}(当然代码会更复杂一些)。

\section{小结}

由于\textbf{卷积神经网络}的复杂性,我们写出了整个系列目前为止最长的一篇文章,相信读者也和作者一样累的要死。\textbf{卷积神经网络}是深度学习最重要的工具(我犹豫要不要写上『之一』呢),付出一些辛苦去理解它也是值得的。如果您真正理解了本文的内容,相当于迈过了入门深度学习最重要的一到门槛。在下一篇文章中,我们介绍深度学习另外一种非常重要的工具:\textbf{循环神经网络},届时我们的系列文章也将完成过半。每篇文章都是一个过滤器,对于坚持到这里的读者们,入门深度学习曙光已现,加油。







\chapter{循环神经网络}\label{chap:Rnn}

\begin{introduction}
	\item 语言模型~\ref{Rnn:1}
	\item 循环神经网络是啥~\ref{Rnn:2}
	\item 基本循环神经网络~\ref{Rnn:3}
	\item 双向循环神经网络~\ref{Rnn:4}
	\item 深度循环神经网络~\ref{Rnn:5}
	\item 循环神经网络的训练~\ref{Rnn:6}
	\item 训练算法:BPTT~\ref{Rnn:7}
	\item 梯度爆炸和消失问题~\ref{Rnn:8}
	\item RNN的应用:语言模型~\ref{Rnn:9}
	\item 向量化~\ref{Rnn:10}
	\item Softmax层~\ref{Rnn:11}
	\item 语言模型的训练~\ref{Rnn:12}
	\item 交叉熵误差~\ref{Rnn:13}
	\item 编程实战:RNN的实现~\ref{Rnn:14}
\end{introduction}


在前面的文章系列文章中,我们介绍了全连接神经网络和卷积神经网络,以及它们的训练和使用。他们都只能单独的取处理一个个的输入,前一个输入和后一个输入是完全没有关系的。但是,某些任务需要能够更好的处理\textbf{序列}的信息,即前面的输入和后面的输入是有关系的。比如,当我们在理解一句话意思时,孤立的理解这句话的每个词是不够的,我们需要处理这些词连接起来的整个\textbf{序列};当我们处理视频的时候,我们也不能只单独的去分析每一帧,而要分析这些帧连接起来的整个\textbf{序列}。这时,就需要用到深度学习领域中另一类非常重要神经网络:\textbf{循环神经网络(Recurrent Neural Network)}。RNN种类很多,也比较绕脑子。不过读者不用担心,本文将一如既往的对复杂的东西剥茧抽丝,帮助您理解RNNs以及它的训练算法,并动手实现一个\textbf{循环神经网络}。


\section{语言模型}\label{Rnn:1}


RNN是在\textbf{自然语言处理}领域中最先被用起来的,比如,RNN可以为\textbf{语言模型}来建模。那么,什么是语言模型呢?

我们可以和电脑玩一个游戏,我们写出一个句子前面的一些词,然后,让电脑帮我们写下接下来的一个词。比如下面这句:
\begin{lstlisting}[numbers=none]
    我昨天上学迟到了,老师批评了 _ _ _ _。
\end{lstlisting}

我们给电脑展示了这句话前面这些词,然后,让电脑写下接下来的一个词。在这个例子中,接下来的这个词最有可能是『我』,而不太可能是『小明』,甚至是『吃饭』。

\textbf{语言模型}就是这样的东西:给定一个一句话前面的部分,预测接下来最有可能的一个词是什么。

\textbf{语言模型}是对一种语言的特征进行建模,它有很多很多用处。比如在语音转文本(STT)的应用中,声学模型输出的结果,往往是若干个可能的候选词,这时候就需要\textbf{语言模型}来从这些候选词中选择一个最可能的。当然,它同样也可以用在图像到文本的识别中(OCR)。

使用RNN之前,语言模型主要是采用N-Gram。N可以是一个自然数,比如2或者3。它的含义是,假设一个词出现的概率只与前面N个词相关。我们以2-Gram为例。首先,对前面的一句话进行切词:
\begin{lstlisting}[numbers=none]
    我 昨天 上学 迟到 了 ,老师 批评 了 _ _ _ _。
\end{lstlisting}


如果用2-Gram进行建模,那么电脑在预测的时候,只会看到前面的『了』,然后,电脑会在语料库中,搜索『了』后面最可能的一个词。不管最后电脑选的是不是『我』,我们都知道这个模型是不靠谱的,因为『了』前面说了那么一大堆实际上是没有用到的。如果是3-Gram模型呢,会搜索『批评了』后面最可能的词,感觉上比2-Gram靠谱了不少,但还是远远不够的。因为这句话最关键的信息『我』,远在9个词之前!

现在读者可能会想,可以提升继续提升N的值呀,比如4-Gram、5-Gram.......。实际上,这个想法是没有实用性的。因为我们想处理任意长度的句子,N设为多少都不合适;另外,模型的大小和N的关系是指数级的,4-Gram模型就会占用海量的存储空间。

所以,该轮到RNN出场了,RNN理论上可以往前看(往后看)任意多个词。

\section{循环神经网络是啥}\label{Rnn:2}

循环神经网络种类繁多,我们先从最简单的基本循环神经网络开始吧。

\subsection{基本循环神经网络}\label{Rnn:3}

图\ref{fig:Rnn1}是一个简单的循环神经网络如,它由输入层、一个隐藏层和一个输出层组成:

\begin{figure}[!h]
	\centering
	\includegraphics[width=0.15\textwidth]{Rnn1.jpg}
	\caption{简单的循环神经网络}
	\label{fig:Rnn1}
\end{figure}

纳尼?!相信第一次看到这个玩意的读者内心和我一样是崩溃的。因为\textbf{循环神经网络}实在是太难画出来了,网上所有大神们都不得不用了这种抽象艺术手法。不过,静下心来仔细看看的话,其实也是很好理解的。如果把上面有W的那个带箭头的圈去掉,它就变成了最普通的\textbf{全连接神经网络}。$x$是一个向量,它表示\textbf{输入层}的值(这里面没有画出来表示神经元节点的圆圈);$s$是一个向量,它表示\textbf{隐藏层}的值(这里隐藏层面画了一个节点,你也可以想象这一层其实是多个节点,节点数与向量$s$的维度相同);$U$是输入层到隐藏层的\textbf{权重矩阵}(读者可以回到第\ref{chap:Bp}章神经网络和反向传播算法,看看我们是怎样用矩阵来表示\textbf{全连接神经网络}的计算的);$o$也是一个向量,它表示\textbf{输出层}的值;$V$是隐藏层到输出层的\textbf{权重矩阵}。那么,现在我们来看看$W$是什么。\textbf{循环神经网络}的\textbf{隐藏层}的值$s$不仅仅取决于当前这次的输入$x$,还取决于上一次\textbf{隐藏层}的值$s$。\textbf{权重矩阵}
$W$就是\textbf{隐藏层}上一次的值作为这一次的输入的权重。

如果我们把上面的图展开,\textbf{循环神经网络}也可以画成图\ref{fig:Rnn2}这个样子:

\begin{figure}[!h]
	\centering
	\includegraphics[width=0.8\textwidth]{Rnn2.jpg}
	\caption{循环神经网络}
	\label{fig:Rnn2}
\end{figure}

现在看上去就比较清楚了,这个网络在t时刻接收到输入\({x}_{t}\)之后,隐藏层的值是\({s}_{t}\),输出值是\({o}_{t}\)。关键一点是,\({s}_t\)的值不仅仅取决于\({x}_t\),还取决于\({s}_{t-1}\)。我们可以用下面的公式来表示\textbf{循环神经网络}的计算方法:
\begin{align}
	{o}_t & =g(V{s}_t)\label{eq:Rnn1}            \\
	{s}_t & =f(U{x}_t+W{s}_{t-1})\label{eq:Rnn2}
\end{align}


公式 \ref{eq:Rnn1} 是\textbf{输出层}的计算公式,输出层是一个\textbf{全连接层},也就是它的每个节点都和隐藏层的每个节点相连。$V$是输出层的\textbf{权重矩阵},$g$是\textbf{激活函数}。公式 \ref{eq:Rnn2} 是隐藏层的计算公式,它是\textbf{循环层}。$U$是输入$x$的权重矩阵,$W$是上一次的值\({s}_{t-1}\)作为这一次的输入的\textbf{权重矩阵},$f$是\textbf{激活函数}。

从上面的公式我们可以看出,\textbf{循环层}和\textbf{全连接层}的区别就是\textbf{循环层}多了一个\textbf{权重矩阵}$W$。

如果反复把公式\ref{eq:Rnn2}带入到公式\ref{eq:Rnn1},我们将得到:
\begin{align*}
	{o}_t & =g(V{s}_t)=Vf(U{x}_t+W{s}_{t-1})= g\Big(Vf\big(U{x}_t+Wf(U{x}_{t-1}+W{s}_{t-2})\big)\Big)             \\
	      & = g\Bigg(Vf\Big(U{x}_t+Wf\big(U{x}_{t-1}+Wf(U{x}_{t-2}+W{s}_{t-3})\big)\Big)\Bigg)                    \\
	      & = g\left(Vf\Bigg(U{x}_t+Wf\Big(U{x}_{t-1}+Wf\big(U{x}_{t-2}+Wf(U{x}_{t-3}+...)\big)\Big)\Bigg)\right)
\end{align*}


从上面可以看出,\textbf{循环神经网络}的输出值\(o_t\),是受前面历次输入值\({x}_t\)、\({x}_{t-1}\)、\({x}_{t-2}\)、\({x}_{t-3}\)、...影响的,这就是为什么\textbf{循环神经网络}可以往前看任意多个\textbf{输入值}的原因。




\subsection{双向循环神经网络}\label{Rnn:4}

对于\textbf{语言模型}来说,很多时候光看前面的词是不够的,比如下面这句话:
\begin{lstlisting}[numbers=none]
    我的手机坏了,我打算 _ _ _ _ 一部新手机。
\end{lstlisting}


可以想象,如果我们只看横线前面的词,手机坏了,那么我是打算修一修?换一部新的?还是大哭一场?这些都是无法确定的。但如果我们也看到了横线后面的词是『一部新手机』,那么,横线上的词填『买』的概率就大得多了。

在上一小节中的\textbf{基本循环神经网络}是无法对此进行建模的,因此,我们需要\textbf{双向循环神经网络},如图\ref{fig:Rnn3}所示。

\begin{figure}[!h]
	\centering
	\includegraphics[width=0.8\textwidth]{Rnn3.png}
	\caption{双向循环神经网络}
	\label{fig:Rnn3}
\end{figure}

当遇到这种从未来穿越回来的场景时,难免处于懵逼的状态。不过我们还是可以用屡试不爽的老办法:先分析一个特殊场景,然后再总结一般规律。我们先考虑图\ref{fig:Rnn3}中,\({y}_2\)的计算。

从图\ref{fig:Rnn3}可以看出,\textbf{双向卷积神经网络}的隐藏层要保存两个值,一个$A$参与正向计算,另一个值$A'$参与反向计算。最终的输出值\({y}_2\)取决于\(A_2\)和\(A_2'\)。其计算方法为:
\[
	{y}_2=g(VA_2+V'A_2')
\]

\(A_2\)和\(A_2'\)则分别计算:
\begin{align*}
	A_2  & =f(WA_1+U{x}_2)    \\
	A_2' & =f(W'A_3'+U'{x}_2)
\end{align*}


现在,我们已经可以看出一般的规律:正向计算时,隐藏层的值\({s}_t\)与\({s}_{t-1}\)有关;反向计算时,隐藏层的值\({s}_t'\)与\({s}_{t+1}'\)有关;最终的输出取决于正向和反向计算的\textbf{加和}。现在,我们仿照公式\ref{eq:Rnn1}和\ref{eq:Rnn2},写出双向循环神经网络的计算方法:
\begin{align*}
	{o}_t  & =g(V{s}_t+V'{s}_t')      \\
	{s}_t  & =f(U{x}_t+W{s}_{t-1})    \\
	{s}_t' & =f(U'{x}_t+W'{s}_{t+1}')
\end{align*}


从上面三个公式我们可以看到,正向计算和反向计算\textbf{不共享权重},也就是说$U$和$U'$、$W$和$W'$、$V$和$V'$都是不同的\textbf{权重矩阵}。




\subsection{深度循环神经网络}\label{Rnn:5}
前面我们介绍的\textbf{循环神经网络}只有一个隐藏层,我们当然也可以堆叠两个以上的隐藏层,这样就得到了\textbf{深度循环神经网络}。如图\ref{fig:Rnn4}所示。


我们把第i个隐藏层的值表示为\({s}_t^{(i)}\)、\({s}_t'^{(i)}\),则\textbf{深度循环神经网络}的计算方式可以表示为:
\begin{align*}
	{o}_t        & =g(V^{(i)}{s}_t^{(i)}+V'^{(i)}{s}_t'^{(i)})   \\
	{s}_t^{(i)}  & =f(U^{(i)}{s}_t^{(i-1)}+W^{(i)}{s}_{t-1})     \\
	{s}_t'^{(i)} & =f(U'^{(i)}{s}_t'^{(i-1)}+W'^{(i)}{s}_{t+1}') \\
	...                                                          \\
	{s}_t^{(1)}  & =f(U^{(1)}{x}_t+W^{(1)}{s}_{t-1})             \\
	{s}_t'^{(1)} & =f(U'^{(1)}{x}_t+W'^{(1)}{s}_{t+1}')
\end{align*}


\begin{figure}[!htbp]
	\centering
	\includegraphics[width=0.4\textwidth]{Rnn4.png}
	\caption{深度循环神经网络}
	\label{fig:Rnn4}
\end{figure}


\section{循环神经网络的训练}\label{Rnn:6}

\subsection{循环神经网络的训练算法:BPTT}\label{Rnn:7}

BPTT算法是针对\textbf{循环层}的训练算法,它的基本原理和BP算法是一样的,也包含同样的三个步骤:

\begin{enumerate}
	\item
	      前向计算每个神经元的输出值;
	\item
	      反向计算每个神经元的\textbf{误差项} \(\delta_j\) 值,它是误差函数$E$对神经元$j$的\textbf{加权输入}\(net_j\)的偏导数;
	\item
	      计算每个权重的梯度。
\end{enumerate}

最后再用\textbf{随机梯度下降}算法更新权重。

循环层如图\ref{fig:Rnn5}所示:

\begin{figure}[!h]
	\centering
	\includegraphics[width=0.6\textwidth]{Rnn5.png}
	\caption{循环层}
	\label{fig:Rnn5}
\end{figure}

\textbf{前向计算}

使用前面的公式\ref{eq:Rnn2}对循环层进行前向计算:
\[
	{s}_t=f(U{x}_t+W{s}_{t-1})
\]

注意,上面的\({s}_t\)、\({x}_t\)、\({s}_{t-1}\)都是向量,用\textbf{黑体字母}表示;而$U$、$V$是\textbf{矩阵},用大写字母表示。\textbf{向量的下标}表示\textbf{时刻},例如,\({s}_t\)表示在$t$时刻向量$s$的值。

我们假设输入向量$x$的维度是$m$,输出向量$s$的维度是$n$,则矩阵$U$的维度是\(n\times m\),矩阵$W$的维度是\(n\times n\)。下面是上式展开成矩阵的样子,看起来更直观一些:
\begin{align*}
	\begin{bmatrix}
		s_1^t  \\
		s_2^t  \\
		\vdots \\
		s_n^t  \\
	\end{bmatrix}=f\left(
	\begin{bmatrix}
		u_{11} u_{12} ... u_{1m} \\
		u_{21} u_{22} ... u_{2m} \\
		\vdots                   \\
		u_{n1} u_{n2} ... u_{nm} \\
	\end{bmatrix}
	\begin{bmatrix}
		x_1^t  \\
		x_2^t  \\
		\vdots \\
		x_m^t  \\
	\end{bmatrix}+
	\begin{bmatrix}
		w_{11} w_{12} ... w_{1n} \\
		w_{21} w_{22} ... w_{2n} \\
		\vdots                   \\
		w_{n1} w_{n2} ... w_{nn} \\
	\end{bmatrix}
	\begin{bmatrix}
		s_1^{t-1} \\
		s_2^{t-1} \\
		\vdots    \\
		s_n^{t-1} \\
	\end{bmatrix}\right)
\end{align*}

在这里我们用\textbf{手写体字母}表示向量的一个\textbf{元素},它的下标表示它是这个向量的第几个元素,它的上标表示第几个\textbf{时刻}。例如,\(s_j^t\)表示向量$s$的第$j$个元素在$t$时刻的值。\(u_{ji}\)表示\textbf{输入层}第$i$个神经元到\textbf{循环层}第$j$个神经元的权重。\(w_{ji}\)表示\textbf{循环层}第$t-1$时刻的第$i$个神经元到\textbf{循环层}第$t$个时刻的第$j$个神经元的权重。

\textbf{误差项的计算}

BTPP算法将第$l$层$t$时刻的\textbf{误差项}\(\delta_t^l\)值沿两个方向传播,一个方向是其传递到上一层网络,得到\(\delta_t^{l-1}\),这部分只和权重矩阵$U$有关;另一个是方向是将其沿时间线传递到初始\(t_1\)时刻,得到\(\delta_1^l\),这部分只和权重矩阵$W$有关。

我们用向量\({net}_t\)表示神经元在$t$时刻的\textbf{加权输入},因为:
\begin{align*}
	{net}_t   & =U{x}_t+W{s}_{t-1} \\
	{s}_{t-1} & =f({net}_{t-1})
\end{align*}


因此:
\begin{align*}
	\frac{\partial{{net}_t}}{\partial{{net}_{t-1}}} & =\frac{\partial{{net}_t}}{\partial{{s}_{t-1}}}\frac{\partial{{s}_{t-1}}}{\partial{{net}_{t-1}}}
\end{align*}


我们用$a$表示列向量,用\({a}^T\)表示行向量。上式的第一项是向量函数对向量求导,其结果为Jacobian矩阵:
\begin{align*}
	\frac{\partial{{net}_t}}{\partial{{s}_{t-1}}} =
	\begin{bmatrix}
		\frac{\partial{net_1^t}}{\partial{s_1^{t-1}}} & \frac{\partial{net_1^t}}{\partial{s_2^{t-1}}} & ... & \frac{\partial{net_1^t}}{\partial{s_n^{t-1}}} \\
		\frac{\partial{net_2^t}}{\partial{s_1^{t-1}}} & \frac{\partial{net_2^t}}{\partial{s_2^{t-1}}} & ... & \frac{\partial{net_2^t}}{\partial{s_n^{t-1}}} \\
		\vdots                                                                                                                                              \\
		\frac{\partial{net_n^t}}{\partial{s_1^{t-1}}} & \frac{\partial{net_n^t}}{\partial{s_2^{t-1}}} & ... & \frac{\partial{net_n^t}}{\partial{s_n^{t-1}}} \\
	\end{bmatrix}
	=\begin{bmatrix}
		w_{11} & w_{12} & ... & w_{1n} \\
		w_{21} & w_{22} & ... & w_{2n} \\
		\vdots                         \\
		w_{n1} & w_{n2} & ... & w_{nn} \\
	\end{bmatrix}=W
\end{align*}


同理,上式第二项也是一个Jacobian矩阵:
\begin{align*}
	\frac{\partial{{s}_{t-1}}}{\partial{{net}_{t-1}}} & =
	\begin{bmatrix}
		\frac{\partial{s_1^{t-1}}}{\partial{net_1^{t-1}}} & \frac{\partial{s_1^{t-1}}}{\partial{net_2^{t-1}}} & ... & \frac{\partial{s_1^{t-1}}}{\partial{net_n^{t-1}}} \\
		\frac{\partial{s_2^{t-1}}}{\partial{net_1^{t-1}}} & \frac{\partial{s_2^{t-1}}}{\partial{net_2^{t-1}}} & ... & \frac{\partial{s_2^{t-1}}}{\partial{net_n^{t-1}}} \\
		\vdots                                                                                                                                                          \\
		\frac{\partial{s_n^{t-1}}}{\partial{net_1^{t-1}}} & \frac{\partial{s_n^{t-1}}}{\partial{net_2^{t-1}}} & ... & \frac{\partial{s_n^{t-1}}}{\partial{net_n^{t-1}}} \\
	\end{bmatrix}                                                      \\
	                                                  & =\begin{bmatrix}
		f'(net_1^{t-1}) & 0               & ... & 0               \\
		0               & f'(net_2^{t-1}) & ... & 0               \\
		\vdots                                                    \\
		0               & 0               & ... & f'(net_n^{t-1}) \\
	\end{bmatrix} \\
	                                                  & =diag[f'({net}_{t-1})]
\end{align*}


其中,$diag({a})$表示根据向量$a$创建一个对角矩阵,即
\[
	diag({a})=\begin{bmatrix}
		a_1 & 0   & ... & 0   \\
		0   & a_2 & ... & 0   \\
		\vdots                \\
		0   & 0   & ... & a_n \\
	\end{bmatrix}
\]

最后,将两项合在一起,可得:
\begin{align*}
	\frac{\partial{{net}_t}}{\partial{{net}_{t-1}}} & =\frac{\partial{{net}_t}}{\partial{{s}_{t-1}}}\frac{\partial{{s}_{t-1}}}{\partial{{net}_{t-1}}}=Wdiag[f'({net}_{t-1})] \\
	                                                & =\begin{bmatrix}
		w_{11}f'(net_1^{t-1}) & w_{12}f'(net_2^{t-1})  & ... & w_{1n}f(net_n^{t-1})   \\
		w_{21}f'(net_1^{t-1}) & w_{22} f'(net_2^{t-1}) & ... & w_{2n}f(net_n^{t-1})   \\
		\vdots                                                                        \\
		w_{n1}f'(net_1^{t-1}) & w_{n2} f'(net_2^{t-1}) & ... & w_{nn} f'(net_n^{t-1}) \\
	\end{bmatrix}
\end{align*}
上式描述了将\(\delta\)沿时间往前传递一个时刻的规律,有了这个规律,我们就可以求得任意时刻$k$的\textbf{误差项}\(\delta_k\):
\begin{align}
	\delta_k^T= & \frac{\partial{E}}{\partial{{net}_k}}=\frac{\partial{E}}{\partial{{net}_t}}\frac{\partial{{net}_t}}{\partial{{net}_k}}=\frac{\partial{E}}{\partial{{net}_t}}\frac{\partial{{net}_t}}{\partial{{net}_{t-1}}}\frac{\partial{{net}_{t-1}}}{\partial{{net}_{t-2}}}...\frac{\partial{{net}_{k+1}}}{\partial{{net}_{k}}}\notag \\
	=           & Wdiag[f'({net}_{t-1})]Wdiag[f'({net}_{t-2})]...Wdiag[f'({net}_{k})]\delta_t^l\notag                                                                                                                                                                                                                                      \\
	=           & \delta_t^T\prod_{i=k}^{t-1}Wdiag[f'({net}_{i})]\label{eq:Rnn3}
\end{align}


公式\ref{eq:Rnn3}就是将误差项沿时间反向传播的算法。

\textbf{循环层}将\textbf{误差项}反向传递到上一层网络,与普通的\textbf{全连接层}是完全一样的,这在第\ref{chap:Bp}章神经网络和反向传播算法中已经详细讲过了,在此仅简要描述一下。

\textbf{循环层}的\textbf{加权输入}\({net}^l\)与上一层的\textbf{加权输入}\({net}^{l-1}\)关系如下:
\begin{align*}
	{net}_t^l=   & U{a}_t^{l-1}+W{s}_{t-1} \\
	{a}_t^{l-1}= & f^{l-1}({net}_t^{l-1})
\end{align*}
上式中\({net}_t^l\)是第$l$层神经元的\textbf{加权输入}(假设第l层是\textbf{循环层});\({net}_t^{l-1}\)是第$l-1$层神经元的\textbf{加权输入};\({a}_t^{l-1}\)是第$l-1$层神经元的输出;\(f^{l-1}\)是第$l-1$层的\textbf{激活函数}。
\[
	\frac{\partial{{net}_t^l}}{\partial{{net}_t^{l-1}}}=\frac{\partial{{net}^l}}{\partial{{a}_t^{l-1}}}\frac{\partial{{a}_t^{l-1}}}{\partial{{net}_t^{l-1}}}=Udiag[f'^{l-1}({net}_t^{l-1})]
\]


所以
\begin{align}
	(\delta_t^{l-1})^T= & \frac{\partial{E}}{\partial{{net}_t^{l-1}}}=\frac{\partial{E}}{\partial{{net}_t^l}}\frac{\partial{{net}_t^l}}{\partial{{net}_t^{l-1}}}\notag \\
	=                   & (\delta_t^l)^TUdiag[f'^{l-1}({net}_t^{l-1})]\label{eq:Rnn4}
\end{align}


公式\ref{eq:Rnn4}就是将误差项传递到上一层算法。

\textbf{权重梯度的计算}

现在,我们终于来到了BPTT算法的最后一步:计算每个权重的梯度。

首先,我们计算误差函数$E$对权重矩阵W的梯度\(\frac{\partial{E}}{\partial{W}}\)。

\begin{figure}[!h]
	\centering
	\includegraphics[width=0.8\textwidth]{Rnn6.png}
	\caption{权重梯度}
	\label{fig:Rnn6}
\end{figure}

图\ref{fig:Rnn6}展示了我们到目前为止,在前两步中已经计算得到的量,包括每个时刻t
\textbf{循环层}的输出值\(s_t\),以及误差项\(\delta_t\)。

回忆一下我们在第\ref{chap:Bp}章神经网络和反向传播算法介绍的全连接网络的权重梯度计算算法:只要知道了任意一个时刻的\textbf{误差项}\(\delta_t\),以及上一个时刻循环层的输出值\({s}_{t-1}\),就可以按照下面的公式求出权重矩阵在$t$时刻的梯度\(\nabla_{Wt}E\):
\begin{equation}
	\label{eq:Rnn5}
	\nabla_{W_t}E=\begin{bmatrix}
		\delta_1^ts_1^{t-1} & \delta_1^ts_2^{t-1} & ... & \delta_1^ts_n^{t-1} \\
		\delta_2^ts_1^{t-1} & \delta_2^ts_2^{t-1} & ... & \delta_2^ts_n^{t-1} \\
		\vdots                                                                \\
		\delta_n^ts_1^{t-1} & \delta_n^ts_2^{t-1} & ... & \delta_n^ts_n^{t-1} \\
	\end{bmatrix}
\end{equation}

在公式\ref{eq:Rnn5}中,\(\delta_i^t\)表示t时刻\textbf{误差项}向量的第$i$个分量;\(s_i^{t-1}\)表示$t-1$时刻\textbf{循环层}第$i$个神经元的输出值。

我们下面可以简单推导一下公式\ref{eq:Rnn5}。

我们知道:
\begin{align*}
	{net}_t=                    & U{x}_t+W{s}_{t-1} \\
	\begin{bmatrix}
		net_1^t \\
		net_2^t \\
		\vdots  \\
		net_n^t \\
	\end{bmatrix}= & U{x}_t+
	\begin{bmatrix}
		w_{11} & w_{12} & ... & w_{1n} \\
		w_{21} & w_{22} & ... & w_{2n} \\
		\vdots                         \\
		w_{n1} & w_{n2} & ... & w_{nn} \\
	\end{bmatrix}
	\begin{bmatrix}
		s_1^{t-1} \\
		s_2^{t-1} \\
		\vdots    \\
		s_n^{t-1} \\
	\end{bmatrix}                      \\
	=                           & U{x}_t+
	\begin{bmatrix}
		w_{11}s_1^{t-1}+w_{12}s_2^{t-1} + \cdots + w_{1n}s_n^{t-1} \\
		w_{21}s_1^{t-1}+w_{22}s_2^{t-1} + \cdots + w_{2n}s_n^{t-1} \\
		\vdots                                                     \\
		w_{n1}s_1^{t-1}+w_{n2}s_2^{t-1} + \cdots + w_{nn}s_n^{t-1} \\
	\end{bmatrix}
\end{align*}


因为对$W$求导与\(U{x}_t\)无关,我们不再考虑。现在,我们考虑对权重项\(w_{ji}\)求导。通过观察上式我们可以看到\(w_{ji}\)只与\(net_j^t\)有关,所以:
\[
	\frac{\partial{E}}{\partial{w_{ji}}}=\frac{\partial{E}}{\partial{net_j^t}}\frac{\partial{net_j^t}}{\partial{w_{ji}}}=\delta_j^ts_i^{t-1}
\]


按照上面的规律就可以生成公式\ref{eq:Rnn5}里面的矩阵。

我们已经求得了权重矩阵$W$在$t$时刻的梯度\(\nabla_{Wt}E\),最终的梯度\(\nabla_WE\)是各个时刻的梯度\textbf{之和}:
\begin{align}
	\nabla_WE= & \sum_{i=1}^t\nabla_{W_i}E\notag \\
	=          & \begin{bmatrix}
		\delta_1^ts_1^{t-1} & \delta_1^ts_2^{t-1} & ... & \delta_1^ts_n^{t-1} \\
		\delta_2^ts_1^{t-1} & \delta_2^ts_2^{t-1} & ... & \delta_2^ts_n^{t-1} \\
		\vdots                                                                \\
		\delta_n^ts_1^{t-1} & \delta_n^ts_2^{t-1} & ... & \delta_n^ts_n^{t-1} \\
	\end{bmatrix}+...+
	\begin{bmatrix}
		\delta_1^1s_1^0 & \delta_1^1s_2^0 & ... & \delta_1^1s_n^0 \\
		\delta_2^1s_1^0 & \delta_2^1s_2^0 & ... & \delta_2^1s_n^0 \\
		\vdots                                                    \\
		\delta_n^1s_1^0 & \delta_n^1s_2^0 & ... & \delta_n^1s_n^0 \\
	\end{bmatrix}\label{eq:Rnn6}
\end{align}

公式\ref{eq:Rnn6}就是计算\textbf{循环层}权重矩阵$W$的梯度的公式。

\textbf{——数学公式超高能预警——}

前面已经介绍了\(\nabla_WE\)的计算方法,看上去还是比较直观的。然而,读者也许会困惑,为什么最终的梯度是各个时刻的梯度\textbf{之和}呢?我们前面只是直接用了这个结论,实际上这里面是有道理的,只是这个数学推导比较绕脑子。感兴趣的同学可以仔细阅读接下来这一段,它用到了矩阵对矩阵求导、张量与向量相乘运算的一些法则。

我们还是从这个式子开始:
\[
	{net}_t=U{x}_t+Wf({net}_{t-1})
\]

因为\(U{x}_t\)与$W$完全无关,我们把它看做常量。现在,考虑第一个式子加号右边的部分,因为$W$和\(f({net}_{t-1})\)都是$W$的函数,因此我们要用到大学里面都学过的导数乘法运算:
\[
	(uv)'=u'v+uv'
\]

因此,上面第一个式子写成:
\[
	\frac{\partial{{net}_t}}{\partial{W}}=\frac{\partial{W}}{\partial{W}}f({net}_{t-1})+W\frac{\partial{f({net}_{t-1})}}{\partial{W}}\\
\]

我们最终需要计算的是\(\nabla_WE\):
\begin{align}
	\nabla_WE= & \frac{\partial{E}}{\partial{W}}=\frac{\partial{E}}{\partial{{net}_t}}\frac{\partial{{net}_t}}{\partial{W}}\notag                \\
	=          & \delta_t^T\frac{\partial{W}}{\partial{W}}f({net}_{t-1})+ \delta_t^TW\frac{\partial{f({net}_{t-1})}}{\partial{W}}\label{eq:Rnn7}
\end{align}

我们先计算公式\ref{eq:Rnn7}加号左边的部分。\(\frac{\partial{W}}{\partial{W}}\)是\textbf{矩阵对矩阵求导},其结果是一个四维\textbf{张量(tensor)},如下所示:

\begin{align*}
	\frac{\partial{W}}{\partial{W}}= &
	\begin{bmatrix}
		\frac{\partial{w_{11}}}{\partial{W}} & \frac{\partial{w_{12}}}{\partial{W}} & ... & \frac{\partial{w_{1n}}}{\partial{W}} \\
		\frac{\partial{w_{21}}}{\partial{W}} & \frac{\partial{w_{22}}}{\partial{W}} & ... & \frac{\partial{w_{2n}}}{\partial{W}} \\\vdots \\
		\frac{\partial{w_{n1}}}{\partial{W}} & \frac{\partial{w_{n2}}}{\partial{W}} & ... & \frac{\partial{w_{nn}}}{\partial{W}} \\
	\end{bmatrix}         \\
	=                                &
	\begin{bmatrix}
		\begin{bmatrix}
			\frac{\partial{w_{11}}}{\partial{w_{11}}} & \frac{\partial{w_{11}}}{\partial{w_{12}}} & ... & \frac{\partial{w_{11}}}{\partial{_{1n}}} \\
			\frac{\partial{w_{11}}}{\partial{w_{21}}} & \frac{\partial{w_{11}}}{\partial{w_{22}}} & ... & \frac{\partial{w_{11}}}{\partial{_{2n}}} \\\vdots\\
			\frac{\partial{w_{11}}}{\partial{w_{n1}}} & \frac{\partial{w_{11}}}{\partial{w_{n2}}} & ... & \frac{\partial{w_{11}}}{\partial{_{nn}}} \\
		\end{bmatrix} &
		\begin{bmatrix}
			\frac{\partial{w_{12}}}{\partial{w_{11}}} & \frac{\partial{w_{12}}}{\partial{w_{12}}} & ... & \frac{\partial{w_{12}}}{\partial{_{1n}}} \\
			\frac{\partial{w_{12}}}{\partial{w_{21}}} & \frac{\partial{w_{12}}}{\partial{w_{22}}} & ... & \frac{\partial{w_{12}}}{\partial{_{2n}}} \\\vdots\\
			\frac{\partial{w_{12}}}{\partial{w_{n1}}} & \frac{\partial{w_{12}}}{\partial{w_{n2}}} & ... & \frac{\partial{w_{12}}}{\partial{_{nn}}} \\
		\end{bmatrix} & ... \\\vdots\\
	\end{bmatrix}         \\
	=                                &
	\begin{bmatrix}
		\begin{bmatrix}
			1 & 0 & ... & 0 \\
			0 & 0 & ... & 0 \\
			\vdots          \\
			0 & 0 & ... & 0 \\
		\end{bmatrix} &
		\begin{bmatrix}
			0 & 1 & ... & 0 \\
			0 & 0 & ... & 0 \\
			\vdots          \\
			0 & 0 & ... & 0 \\
		\end{bmatrix} & ... \\
		\vdots                           \\
	\end{bmatrix}
\end{align*}


接下来,我们知道\(s_{t-1}=f({{net}_{t-1}})\),它是一个\textbf{列向量}。我们让上面的四维张量与这个向量相乘,得到了一个三维张量,再左乘行向量\(\delta_t^T\),最终得到一个矩阵:

\begin{align*}
	\delta_t^T\frac{\partial{W}}{\partial{W}}f({{net}_{t-1}})= & \delta_t^T\frac{\partial{W}}{\partial{W}}{{s}_{t-1}}=\delta_t^T
	\begin{bmatrix}
		\begin{bmatrix}
			1 & 0 & ... & 0 \\
			0 & 0 & ... & 0 \\
			\vdots          \\
			0 & 0 & ... & 0 \\
		\end{bmatrix} &
		\begin{bmatrix}
			0 & 1 & ... & 0 \\
			0 & 0 & ... & 0 \\
			\vdots          \\
			0 & 0 & ... & 0 \\
		\end{bmatrix} & ... \\
		\vdots                           \\
	\end{bmatrix}
	\begin{bmatrix}
		s_1^{t-1} \\
		s_2^{t-1} \\
		\vdots    \\
		s_n^{t-1} \\
	\end{bmatrix}                                                                                                   \\
	=                                                          & \delta_t^T
	\begin{bmatrix}
		\begin{bmatrix}
			s_1^{t-1} \\
			0         \\
			\vdots    \\
			0         \\
		\end{bmatrix} &
		\begin{bmatrix}
			s_2^{t-1} \\
			0         \\
			\vdots    \\
			0         \\
		\end{bmatrix} & ... \\
		\vdots                           \\
	\end{bmatrix}=
	\begin{bmatrix}
		\delta_1^t & \delta_2^t & ... & \delta_n^t
	\end{bmatrix}
	\begin{bmatrix}
		\begin{bmatrix}
			s_1^{t-1} \\
			0         \\
			\vdots    \\
			0         \\
		\end{bmatrix} &
		\begin{bmatrix}
			s_2^{t-1} \\
			0         \\
			\vdots    \\
			0         \\
		\end{bmatrix} & ... \\
		\vdots                           \\
	\end{bmatrix}                                                                                                   \\
	=                                                          &
	\begin{bmatrix}
		\delta_1^ts_1^{t-1} & \delta_1^ts_2^{t-1} & ... & \delta_1^ts_n^{t-1} \\
		\delta_2^ts_1^{t-1} & \delta_2^ts_2^{t-1} & ... & \delta_2^ts_n^{t-1} \\
		\vdots                                                                \\
		\delta_n^ts_1^{t-1} & \delta_n^ts_2^{t-1} & ... & \delta_n^ts_n^{t-1} \\
	\end{bmatrix}=\nabla_{Wt}E
\end{align*}


接下来,我们计算公式\ref{eq:Rnn7}加号右边的部分:
\begin{align*}
	\delta_t^TW\frac{\partial{f({net}_{t-1})}}{\partial{W}}= & \delta_t^TW\frac{\partial{f({net}_{t-1})}}{\partial{{net}_{t-1}}}\frac{\partial{{net}_{t-1}}}{\partial{W}}=\delta_t^TWf'({net}_{t-1})\frac{\partial{{net}_{t-1}}}{\partial{W}} \\
	=                                                        & \delta_t^T\frac{\partial{{net}_t}}{\partial{{net}_{t-1}}}\frac{\partial{{net}_{t-1}}}{\partial{W}}=\delta_{t-1}^T\frac{\partial{{net}_{t-1}}}{\partial{W}}
\end{align*}


于是,我们得到了如下递推公式:
\begin{align*}
	\nabla_WE= & \frac{\partial{E}}{\partial{W}}=\frac{\partial{E}}{\partial{{net}_t}}\frac{\partial{{net}_t}}{\partial{W}}=\nabla_{Wt}E+\delta_{t-1}^T\frac{\partial{{net}_{t-1}}}{\partial{W}} \\
	=          & \nabla_{Wt}E+\nabla_{Wt-1}E+\delta_{t-2}^T\frac{\partial{{net}_{t-2}}}{\partial{W}}                                                                                             \\
	=          & \nabla_{Wt}E+\nabla_{Wt-1}E+...+\nabla_{W1}E                                                                                                                                    \\
	=          & \sum_{k=1}^t\nabla_{Wk}E
\end{align*}

这样,我们就证明了:最终的梯度\(\nabla_WE\)是各个时刻的梯度之和。

\textbf{---数学公式超高能预警解除---}

同权重矩阵$W$类似,我们可以得到权重矩阵$U$的计算方法。

\begin{equation}
	\label{eq:Rnn8}
	\nabla_{U_t}E=\begin{bmatrix}
		\delta_1^tx_1^t & \delta_1^tx_2^t & ... & \delta_1^tx_m^t \\
		\delta_2^tx_1^t & \delta_2^tx_2^t & ... & \delta_2^tx_m^t \\
		\vdots                                                    \\
		\delta_n^tx_1^t & \delta_n^tx_2^t & ... & \delta_n^tx_m^t \\
	\end{bmatrix}
\end{equation}

公式\ref{eq:Rnn8}是误差函数在$t$时刻对权重矩阵$U$的梯度。和权重矩阵$W$一样,最终的梯度也是各个时刻的梯度之和:
\[
	\nabla_UE=\sum_{i=1}^t\nabla_{U_i}E
\]

具体的证明这里就不再赘述了,感兴趣的读者可以练习推导一下。

\subsection{RNN的梯度爆炸和消失问题}\label{Rnn:8}

不幸的是,实践中前面介绍的几种RNNs并不能很好的处理较长的序列。一个主要的原因是,RNN在训练中很容易发生\textbf{梯度爆炸}和\textbf{梯度消失},这导致训练时梯度不能在较长序列中一直传递下去,从而使RNN无法捕捉到长距离的影响。

为什么RNN会产生梯度爆炸和消失问题呢?我们接下来将详细分析一下原因。我们根据公式\ref{eq:Rnn3}可得:
\begin{align*}
	\delta_k^T=             & \delta_t^T\prod_{i=k}^{t-1}Wdiag[f'(\mathrm{net}_{i})]             \\
	\|\delta_k^T\|\leqslant & \|\delta_t^T\|\prod_{i=k}^{t-1}\|W\|\|diag[f'(\mathrm{net}_{i})]\| \\
	\leqslant               & \|\delta_t^T\|(\beta_W\beta_f)^{t-k}
\end{align*}

上式的\(\beta\)定义为矩阵的模的上界。因为上式是一个指数函数,如果$t-k$很大的话(也就是向前看很远的时候),会导致对应的\textbf{误差项}的值增长或缩小的非常快,这样就会导致相应的\textbf{梯度爆炸}和\textbf{梯度消失}问题(取决于\(\beta\)大于1还是小于1)。

通常来说,\textbf{梯度爆炸}更容易处理一些。因为梯度爆炸的时候,我们的程序会收到$NaN$错误。我们也可以设置一个梯度阈值,当梯度超过这个阈值的时候可以直接截取。

\textbf{梯度消失}更难检测,而且也更难处理一些。总的来说,我们有三种方法应对梯度消失问题:

\begin{enumerate}
	\item
	      合理的初始化权重值。初始化权重,使每个神经元尽可能不要取极大或极小值,以躲开梯度消失的区域。
	\item
	      使用relu代替sigmoid和tanh作为激活函数。原理请参考第\ref{chap:Cnn}章卷积神经网络的\ref{Cnn:1}\textbf{激活函数}一节。
	\item
	      使用其他结构的RNNs,比如长短时记忆网络(LTSM)和Gated Recurrent   Unit(GRU),这是最流行的做法。我们将在以后的文章中介绍这两种网络。
\end{enumerate}







\section{RNN的应用:基于RNN的语言模型}\label{Rnn:9}

现在,我们介绍一下基于RNN语言模型。我们首先把词依次输入到循环神经网络中,每输入一个词,循环神经网络就输出截止到目前为止,下一个最可能的词。例如,当我们依次输入:
\begin{lstlisting}[numbers=none]
    我 昨天 上学 迟到 了
\end{lstlisting}

神经网络的输出如图\ref{fig:Rnn7}所示:
\begin{figure}[!h]
	\centering
	\includegraphics[width=0.7\textwidth]{Rnn7.png}
	\caption{神经网络的输出}
	\label{fig:Rnn7}
\end{figure}

其中,s和e是两个特殊的词,分别表示一个序列的开始和结束。

\subsection{向量化}\label{Rnn:10}

我们知道,神经网络的输入和输出都是\textbf{向量},为了让语言模型能够被神经网络处理,我们必须把词表达为向量的形式,这样神经网络才能处理它。

神经网络的输入是\textbf{词},我们可以用下面的步骤对输入进行\textbf{向量化}:

\begin{enumerate}
	\item
	      建立一个包含所有词的词典,每个词在词典里面有一个唯一的编号。
	\item
	      任意一个词都可以用一个$N$维的one-hot向量来表示。其中,$N$是词典中包含的词的个数。假设一个词在词典中的编号是$i$,$v$是表示这个词的向量,\(v_j\)是向量的第$j$个元素,则:
	      \begin{equation*}
		      v_j=\left\{
		      \begin{aligned}
			      1 & \quad j=i       \\
			      0 & \quad otherwise
		      \end{aligned}
		      \right.
	      \end{equation*}
\end{enumerate}


上面这个公式的含义,可以用图\ref{fig:Rnn8}来直观的表示。
\begin{figure}[!h]
	\centering
	\includegraphics[width=0.9\textwidth]{Rnn8.png}
	\caption{向量化}
	\label{fig:Rnn8}
\end{figure}
使用这种向量化方法,我们就得到了一个高维、\textbf{稀疏}的向量(稀疏是指绝大部分元素的值都是0)。处理这样的向量会导致我们的神经网络有很多的参数,带来庞大的计算量。因此,往往会需要使用一些降维方法,将高维的稀疏向量转变为低维的稠密向量。不过这个话题我们就不再这篇文章中讨论了。

语言模型要求的输出是下一个最可能的词,我们可以让循环神经网络计算计算词典中每个词是下一个词的概率,这样,概率最大的词就是下一个最可能的词。因此,神经网络的输出向量也是一个$N$维向量,向量中的每个元素对应着词典中相应的词是下一个词的概率。如图\ref{fig:Rnn9}所示:

\begin{figure}[!h]
	\centering
	\includegraphics[width=0.9\textwidth]{Rnn9.png}
	\caption{向量化}
	\label{fig:Rnn9}
\end{figure}

\subsection{Softmax层}\label{Rnn:11}

前面提到,\textbf{语言模型}是对下一个词出现的\textbf{概率}进行建模。那么,怎样让神经网络输出概率呢?方法就是用softmax层作为神经网络的输出层。

我们先来看一下softmax函数的定义:
\[
	g(z_i)=\frac{e^{z_i}}{\sum_{k}e^{z_k}}
\]

这个公式看起来可能很晕,我们举一个例子。Softmax层如图\ref{fig:Rnn10}所示,
从图\ref{fig:Rnn10}我们可以看到,softmax layer的输入是一个向量,输出也是一个向量,两个向量的维度是一样的(在这个例子里面是4)。输入向量$x=[1, 2, 3, 4]$经过softmax层之后,经过上面的softmax函数计算,转变为输出向量$y=[0.03, 0.09, 0.24, 0.64]$。计算过程为:
\begin{align*}
	y_1 & =\frac{e^{x_1}}{\sum_{k}e^{x_k}}=\frac{e^1}{e^1+e^2+e^3+e^4}=0.03 \\
	y_2 & =\frac{e^2}{e^1+e^2+e^3+e^4}=0.09                                 \\
	y_3 & =\frac{e^3}{e^1+e^2+e^3+e^4}=0.24                                 \\
	y_4 & =\frac{e^4}{e^1+e^2+e^3+e^4}=0.64
\end{align*}

\begin{figure}[!h]
	\centering
	\includegraphics[width=0.5\textwidth]{Rnn10.png}
	\caption{Softmax层}
	\label{fig:Rnn10}
\end{figure}

我们来看看输出向量$y$的特征:

\begin{enumerate}
	\item
	      每一项为取值为0-1之间的正数;
	\item
	      所有项的总和是1。
\end{enumerate}

我们不难发现,这些特征和\textbf{概率}的特征是一样的,因此我们可以把它们看做是概率。对于\textbf{语言模型}来说,我们可以认为模型预测下一个词是词典中第一个词的概率是0.03,是词典中第二个词的概率是0.09,以此类推。

\subsection{语言模型的训练}\label{Rnn:12}

可以使用\textbf{监督学习}的方法对语言模型进行训练,首先,需要准备训练数据集。接下来,我们介绍怎样把语料
\begin{lstlisting}[numbers=none]
    我 昨天 上学 迟到 了
\end{lstlisting}

转换成语言模型的训练数据集。

首先,我们获取\textbf{输入-标签}对:

\begin{table}[!h]
	\centering
	\setlength{\tabcolsep}{10mm}
	\caption{输入-标签}
	\begin{tabular}{cc}
		\hline
		输入 & 标签 \\ \hline
		s    & 我   \\
		我   & 昨天 \\
		昨天 & 上学 \\
		上学 & 迟到 \\
		迟到 & 了   \\
		了   & e    \\ \hline
	\end{tabular}
\end{table}


然后,使用前面介绍过的\textbf{向量化}方法,对输入$x$和标签$y$进行\textbf{向量化}。这里面有意思的是,对标签$y$进行向量化,其结果也是一个one-hot向量。例如,我们对标签『我』进行向量化,得到的向量中,只有第2019个元素的值是1,其他位置的元素的值都是0。它的含义就是下一个词是『我』的概率是1,是其它词的概率都是0。

最后,我们使用\textbf{交叉熵误差函数}作为优化目标,对模型进行优化。

在实际工程中,我们可以使用大量的语料来对模型进行训练,获取训练数据和训练的方法都是相同的。

\subsection{交叉熵误差}\label{Rnn:13}

一般来说,当神经网络的输出层是softmax层时,对应的误差函数$S$通常选择交叉熵误差函数,其定义如下:
\[
	L(y,o)=-\frac{1}{N}\sum_{n\in{N}}{y_nlogo_n}
\]

在上式中,$N$是训练样本的个数,向量\(y_n\)是样本的标记,向量\(o_n\)是网络的输出。标记\(y_n\)是一个one-hot向量,例如\(y_1=[1,0,0,0]\),如果网络的输出\(o=[0.03,0.09,0.24,0.64]\),那么,交叉熵误差是(假设只有一个训练样本,即N=1):
\begin{align*}
	L & =-\frac{1}{N}\sum_{n\in{N}}{y_nlogo_n}=-y_1logo_1 \\
	  & =-(1*log0.03+0*log0.09+0*log0.24+0*log0.64)=3.51
\end{align*}


我们当然可以选择其他函数作为我们的误差函数,比如最小平方误差函数(MSE)。不过对概率进行建模时,选择交叉熵误差函数更make sense。具体原因,感兴趣的读者请阅读(\url{https://jamesmccaffrey.wordpress.com/2011/12/17/neural-network-classification-categorical-data-softmax-activation-and-cross-entropy-error/})。




\section{编程实战:RNN的实现}\label{Rnn:14}

\begin{note}
	完整代码请参考GitHub: \url{https://github.com/hanbt/learn_dl/blob/master/rnn.py}
	(python2.7)
\end{note}

为了加深我们对前面介绍的知识的理解,我们来动手实现一个RNN层。我们复用了第\ref{chap:Cnn}章卷积神经网络中的一些代码,所以先把它们导入进来。
\begin{lstlisting}
import numpy as np
from cnn import ReluActivator, IdentityActivator, element_wise_op
\end{lstlisting}

我们用RecurrentLayer类来实现一个\textbf{循环层}。下面的代码是初始化一个循环层,可以在构造函数中设置卷积层的超参数。我们注意到,循环层有两个权重数组,U和W。
\begin{lstlisting}
class RecurrentLayer(object):
    def __init__(self, input_width, state_width,
                 activator, learning_rate):
        self.input_width = input_width
        self.state_width = state_width
        self.activator = activator
        self.learning_rate = learning_rate
        self.times = 0       # 当前时刻初始化为t0
        self.state_list = [] # 保存各个时刻的state
        self.state_list.append(np.zeros(
            (state_width, 1)))           # 初始化s0
        self.U = np.random.uniform(-1e-4, 1e-4,
            (state_width, input_width))  # 初始化U
        self.W = np.random.uniform(-1e-4, 1e-4,
            (state_width, state_width))  # 初始化W
\end{lstlisting}

在forward方法中,实现循环层的前向计算,这部分比较简单。
\begin{lstlisting}
    def forward(self, input_array):
        '''
        根据『式2』进行前向计算
        '''
        self.times += 1
        state = (np.dot(self.U, input_array) +
                 np.dot(self.W, self.state_list[-1]))
        element_wise_op(state, self.activator.forward)
        self.state_list.append(state)
\end{lstlisting}

在backword方法中,实现BPTT算法。
\begin{lstlisting}
    def backward(self, sensitivity_array, 
                 activator):
        '''
        实现BPTT算法
        '''
        self.calc_delta(sensitivity_array, activator)
        self.calc_gradient()
    def calc_delta(self, sensitivity_array, activator):
        self.delta_list = []  # 用来保存各个时刻的误差项
        for i in range(self.times):
            self.delta_list.append(np.zeros(
                (self.state_width, 1)))
        self.delta_list.append(sensitivity_array)
        # 迭代计算每个时刻的误差项
        for k in range(self.times - 1, 0, -1):
            self.calc_delta_k(k, activator)
    def calc_delta_k(self, k, activator):
        '''
        根据k+1时刻的delta计算k时刻的delta
        '''
        state = self.state_list[k+1].copy()
        element_wise_op(self.state_list[k+1],
                    activator.backward)
        self.delta_list[k] = np.dot(
            np.dot(self.delta_list[k+1].T, self.W),
            np.diag(state[:,0])).T
    def calc_gradient(self):
        self.gradient_list = [] # 保存各个时刻的权重梯度
        for t in range(self.times + 1):
            self.gradient_list.append(np.zeros(
                (self.state_width, self.state_width)))
        for t in range(self.times, 0, -1):
            self.calc_gradient_t(t)
        # 实际的梯度是各个时刻梯度之和
        self.gradient = reduce(
            lambda a, b: a + b, self.gradient_list,
            self.gradient_list[0]) # [0]被初始化为0且没有被修改过
    def calc_gradient_t(self, t):
        '''
        计算每个时刻t权重的梯度
        '''
        gradient = np.dot(self.delta_list[t],
            self.state_list[t-1].T)
        self.gradient_list[t] = gradient
\end{lstlisting}

有意思的是,BPTT算法虽然数学推导的过程很麻烦,但是写成代码却并不复杂。

在update方法中,实现梯度下降算法。
\begin{lstlisting}
    def update(self):
        '''
        按照梯度下降,更新权重
        '''
        self.W -= self.learning_rate * self.gradient
\end{lstlisting}

上面的代码不包含权重U的更新。这部分实际上和全连接神经网络是一样的,留给感兴趣的读者自己来完成吧。

\textbf{循环层}是一个\textbf{带状态}的层,每次forword都会改变循环层的内部状态,这给梯度检查带来了麻烦。因此,我们需要一个reset\_state方法,来重置循环层的内部状态。
\begin{lstlisting}
    def reset_state(self):
        self.times = 0       # 当前时刻初始化为t0
        self.state_list = [] # 保存各个时刻的state
        self.state_list.append(np.zeros(
            (self.state_width, 1)))      # 初始化s0
\end{lstlisting}

最后,是梯度检查的代码。
\begin{lstlisting}
def gradient_check():
    '''
    梯度检查
    '''
    # 设计一个误差函数,取所有节点输出项之和
    error_function = lambda o: o.sum()
    rl = RecurrentLayer(3, 2, IdentityActivator(), 1e-3)
    # 计算forward值
    x, d = data_set()
    rl.forward(x[0])
    rl.forward(x[1])
    # 求取sensitivity map
    sensitivity_array = np.ones(rl.state_list[-1].shape,
                                dtype=np.float64)
    # 计算梯度
    rl.backward(sensitivity_array, IdentityActivator())
    # 检查梯度
    epsilon = 10e-4
    for i in range(rl.W.shape[0]):
        for j in range(rl.W.shape[1]):
            rl.W[i,j] += epsilon
            rl.reset_state()
            rl.forward(x[0])
            rl.forward(x[1])
            err1 = error_function(rl.state_list[-1])
            rl.W[i,j] -= 2*epsilon
            rl.reset_state()
            rl.forward(x[0])
            rl.forward(x[1])
            err2 = error_function(rl.state_list[-1])
            expect_grad = (err1 - err2) / (2 * epsilon)
            rl.W[i,j] += epsilon
            print 'weights(%d,%d): expected - actural %f - %f' % (
                i, j, expect_grad, rl.gradient[i,j])
\end{lstlisting}

需要注意,每次计算error之前,都要调用reset\_state方法重置循环层的内部状态。下面是梯度检查的结果,没问题!

\includegraphics[width=0.7\textwidth]{Rnn11.png}


\section{小节}

至此,我们讲完了基本的\textbf{循环神经网络}、它的训练算法:\textbf{BPTT},以及在语言模型上的应用。RNN比较烧脑,相信拿下前几篇文章的读者们搞定这篇文章也不在话下吧!然而,\textbf{循环神经网络}这个话题并没有完结。我们在前面说到过,基本的循环神经网络存在梯度爆炸和梯度消失问题,并不能真正的处理好长距离的依赖(虽然有一些技巧可以减轻这些问题)。事实上,真正得到广泛的应用的是循环神经网络的一个变体:\textbf{长短时记忆网络}。它内部有一些特殊的结构,可以很好的处理长距离的依赖,我们将在下一篇文章中详细的介绍它。现在,让我们稍事休息,准备挑战更为烧脑的\textbf{长短时记忆网络}吧。



\chapter{长短时记忆网络}\label{chap:Lstm}

\begin{introduction}
	\item 长短时记忆网络是啥~\ref{Lstm:1}
	\item 长短时记忆网络的前向计算~\ref{Lstm:2}
	\item 长短时记忆网络的训练~\ref{Lstm:3}
	\item LSTM训练算法框架~\ref{Lstm:4}
	\item 关于公式和符号的说明~\ref{Lstm:5}
	\item 误差项沿时间的反向传递~\ref{Lstm:6}
	\item 将误差项传递到上一层~\ref{Lstm:7}
	\item 权重梯度的计算~\ref{Lstm:8}
	\item 编程实战:长短时记忆网络的实现~\ref{Lstm:9}
	\item 激活函数的实现~\ref{Lstm:10}
	\item LSTM初始化~\ref{Lstm:11}
	\item 前向计算的实现~\ref{Lstm:12}
	\item 反向传播算法的实现~\ref{Lstm:13}
	\item 梯度下降算法的实现~\ref{Lstm:14}
	\item 梯度检查的实现~\ref{Lstm:15}
	\item GRU~\ref{Lstm:16}
\end{introduction}

在上一篇文章中,我们介绍了\textbf{循环神经网络}以及它的训练算法。我们也介绍了\textbf{循环神经网络}很难训练的原因,这导致了它在实际应用中,很难处理长距离的依赖。在本文中,我们将介绍一种改进之后的循环神经网络:\textbf{长短时记忆网络(Long Short Term Memory Network, LSTM)},它成功的解决了原始循环神经网络的缺陷,成为当前最流行的RNN,在语音识别、图片描述、自然语言处理等许多领域中成功应用。但不幸的一面是,\textbf{LSTM}的结构很复杂,因此,我们需要花上一些力气,才能把LSTM以及它的训练算法弄明白。在搞清楚\textbf{LSTM}之后,我们再介绍一种\textbf{LSTM}的变体:\textbf{GRU (Gated Recurrent Unit)}。它的结构比\textbf{LSTM}简单,而效果却和\textbf{LSTM}一样好,因此,它正在逐渐流行起来。最后,我们仍然会动手实现一个\textbf{LSTM}。

\section{长短时记忆网络是啥}\label{Lstm:1}

我们首先了解一下长短时记忆网络产生的背景。回顾一下第\ref{chap:Cnn}章循环神经网络中推导的,误差项沿时间反向传播的公式:
\[
	\delta_k^T=\delta_t^T\prod_{i=k}^{t-1}diag[f'({net}_{i})]W
\]

我们可以根据下面的不等式,来获取\(\delta_k^T\)的模的上界(模可以看做对\(\delta_k^T\)中每一项值的大小的度量):
\begin{align*}
	\|\delta_k^T\|\leqslant & \|\delta_t^T\|\prod_{i=k}^{t-1}\|diag[f'({net}_{i})]\|\|W\| \\
	\leqslant               & \|\delta_t^T\|(\beta_f\beta_W)^{t-k}
\end{align*}

我们可以看到,误差项\(\delta\)从t时刻传递到k时刻,其值的上界是\(\beta_f\beta_w\)的指数函数。\(\beta_f\beta_w\)分别是对角矩阵\(diag[f'({net}_{i})]\)和矩阵W模的上界。显然,除非\(\beta_f\beta_w\)乘积的值位于1附近,否则,当t-k很大时(也就是误差传递很多个时刻时),整个式子的值就会变得极小(当\(\beta_f\beta_w\)乘积小于1)或者极大(当\(\beta_f\beta_w\)乘积大于1),前者就是\textbf{梯度消失},后者就是\textbf{梯度爆炸}。虽然科学家们搞出了很多技巧(比如怎样初始化权重),让\(\beta_f\beta_w\)的值尽可能贴近于1,终究还是难以抵挡指数函数的威力。

\textbf{梯度消失}到底意味着什么?在第\ref{chap:Cnn}章循环神经网络中我们已证明,权重数组W最终的梯度是各个时刻的梯度之和,即:
\begin{align*}
	\nabla_WE & =\sum_{k=1}^t\nabla_{Wk}E=\nabla_{Wt}E+\nabla_{Wt-1}E+\nabla_{Wt-2}E+...+\nabla_{W1}E
\end{align*}

假设某轮训练中,各时刻的梯度以及最终的梯度之和如图\ref{fig:Lstm1}:
\begin{figure}[!h]
	\centering
	\includegraphics[width=0.7\textwidth]{Lstm1.png}
	\caption{梯度}
	\label{fig:Lstm1}
\end{figure}
我们就可以看到,从上图的t-3时刻开始,梯度已经几乎减少到0了。那么,从这个时刻开始再往之前走,得到的梯度(几乎为零)就不会对最终的梯度值有任何贡献,这就相当于无论t-3时刻之前的网络状态h是什么,在训练中都不会对权重数组W的更新产生影响,也就是网络事实上已经忽略了t-3时刻之前的状态。这就是原始RNN无法处理长距离依赖的原因。

既然找到了问题的原因,那么我们就能解决它。从问题的定位到解决,科学家们大概花了7、8年时间。终于有一天,Hochreiter和Schmidhuber两位科学家发明出\textbf{长短时记忆网络},一举解决这个问题。

其实,\textbf{长短时记忆网络}的思路比较简单。原始RNN的隐藏层只有一个状态,即h,它对于短期的输入非常敏感。那么,假如我们再增加一个状态,即c,让它来保存长期的状态,那么问题不就解决了么?如图\ref{fig:Lstm2}所示:
\begin{figure}[!h]
	\centering
	\includegraphics[width=0.35\textwidth]{Lstm2.png}
	\caption{RNN to LSTM}
	\label{fig:Lstm2}
\end{figure}

新增加的状态c,称为\textbf{单元状态(cell state)}。我们把上图按照时间维度展开:
\begin{figure}[!h]
	\centering
	\includegraphics[width=0.7\textwidth]{Lstm3.png}
	\caption{梯度}
	\label{fig:Lstm3}
\end{figure}

图\ref{fig:Lstm3}仅仅是一个示意图,我们可以看出,在t时刻,LSTM的输入有三个:当前时刻网络的输入值\({x}_t\)、上一时刻LSTM的输出值\({h}_{t-1}\)、以及上一时刻的单元状态\({c}_{t-1}\);LSTM的输出有两个:当前时刻LSTM输出值\({h}_t\)、和当前时刻的单元状态\({c}_t\)。注意\({x}\)、\({h}\)、\({c}\)都是\textbf{向量}。

LSTM的关键,就是怎样控制长期状态c。在这里,LSTM的思路是使用三个控制开关。第一个开关,负责控制继续保存长期状态c;第二个开关,负责控制把即时状态输入到长期状态c;第三个开关,负责控制是否把长期状态c作为当前的LSTM的输出。三个开关的作用如图\ref{fig:Lstm4}所示:

\begin{figure}[!h]
	\centering
	\includegraphics[width=0.6\textwidth]{Lstm4.png}
	\caption{梯度}
	\label{fig:Lstm4}
\end{figure}


接下来,我们要描述一下,输出h和单元状态c的具体计算方法。

\section{长短时记忆网络的前向计算}\label{Lstm:2}

前面描述的开关是怎样在算法中实现的呢?这就用到了\textbf{门(gate)}的概念。门实际上就是一层\textbf{全连接层},它的输入是一个向量,输出是一个0到1之间的实数向量。假设W是门的权重向量,\({b}\)是偏置项,那么门可以表示为:
\[
	g({x})=\sigma(W{x}+{b})
\]

门的使用,就是用门的输出向量按元素乘以我们需要控制的那个向量。因为门的输出是0到1之间的实数向量,那么,当门输出为0时,任何向量与之相乘都会得到0向量,这就相当于啥都不能通过;输出为1时,任何向量与之相乘都不会有任何改变,这就相当于啥都可以通过。因为\(\sigma\)(也就是sigmoid函数)的值域是(0,1),所以门的状态都是半开半闭的。

LSTM用两个门来控制单元状态c的内容,一个是\textbf{遗忘门(forget gate)},它决定了上一时刻的单元状态\({c}_{t-1}\)有多少保留到当前时刻\({c}_t\);另一个是\textbf{输入门(input gate)},它决定了当前时刻网络的输入\({x}_t\)有多少保存到单元状态\({c}_t\)。LSTM用\textbf{输出门(output gate)}来控制单元状态\({c}_t\)有多少输出到LSTM的当前输出值\({h}_t\)。

我们先来看一下遗忘门:
\begin{equation}
	\label{eq:Lstm1}
	{f}_t=\sigma(W_f\cdot[{h}_{t-1},{x}_t]+{b}_f)
\end{equation}
上式中,\(W_f\)是遗忘门的权重矩阵,\([{h}_{t-1},{x}_t]\)表示把两个向量连接成一个更长的向量,\({b}_f\)是遗忘门的偏置项,\(\sigma\)是sigmoid函数。如果输入的维度是\(d_x\),隐藏层的维度是\(d_h\),单元状态的维度是\(d_c\)(通常\(d_c=d_h\)),则遗忘门的权重矩阵\(W_f\)维度是\( d_c\times (d_h+d_x)\)。事实上,权重矩阵\(W_f\)都是两个矩阵拼接而成的:一个是\(W_{fh}\),它对应着输入项\({h}_{t-1}\),其维度为\(d_c\times d_h\);一个是\(W_{fx}\),它对应着输入项\({x}_t\),其维度为\(d_c\times d_x\)。\(W_f\)可以写为:
\begin{align*}
	\begin{bmatrix}W_f\end{bmatrix}\begin{bmatrix}{h}_{t-1} \\
		{x}_t\end{bmatrix} & =\begin{bmatrix}W_{fh}&W_{fx}\end{bmatrix}\begin{bmatrix}{h}_{t-1} \\
		{x}_t\end{bmatrix} \\
	                                                    & =W_{fh}{h}_{t-1}+W_{fx}{x}_t
\end{align*}

图\ref{fig:Lstm5}显示了遗忘门的计算:
\begin{figure}[!h]
	\centering
	\includegraphics[width=0.6\textwidth]{Lstm5.png}
	\caption{遗忘门}
	\label{fig:Lstm5}
\end{figure}

接下来看看输入门:
\begin{equation}
	\label{eq:Lstm2}
	{i}_t=\sigma(W_i\cdot[{h}_{t-1},{x}_t]+{b}_i)
\end{equation}
上式中,\(W_i\)是输入门的权重矩阵,\({b}_i\)是输入门的偏置项。图\ref{fig:Lstm6}表示了输入门的计算:
\begin{figure}[!h]
	\centering
	\includegraphics[width=0.6\textwidth]{Lstm6.png}
	\caption{输入门}
	\label{fig:Lstm6}
\end{figure}

接下来,我们计算用于描述当前输入的单元状态\({\tilde{c}}_t\),它是根据上一次的输出和本次输入来计算的:
\begin{equation}
	\label{eq:Lstm3}
	{\tilde{c}}_t=\tanh(W_c\cdot[{h}_{t-1},{x}_t]+{b}_c)
\end{equation}

图\ref{fig:Lstm7}是\({\tilde{c}}_t\)的计算:
\begin{figure}[!h]
	\centering
	\includegraphics[width=0.6\textwidth]{Lstm7.png}
	\caption{\({\tilde{c}}_t\)的计算}
	\label{fig:Lstm7}
\end{figure}
现在,我们计算当前时刻的单元状态\({c}_t\)。它是由上一次的单元状态\({c}_{t-1}\)按元素乘以遗忘门\(f_t\),再用当前输入的单元状态\({\tilde{c}}_t\)按元素乘以输入门\(i_t\),再将两个积加和产生的:
\begin{equation}
	\label{eq:Lstm4}
	{c}_t=f_t\circ{{c}_{t-1}}+i_t\circ{{\tilde{c}}_t}
\end{equation}
符号\(\circ\)表示\textbf{按元素乘}。图\ref{fig:Lstm8}是\({c}_t\)的计算:

\begin{figure}[!h]
	\centering
	\includegraphics[width=0.6\textwidth]{Lstm8.png}
	\caption{\({c}_t\)的计算}
	\label{fig:Lstm8}
\end{figure}

这样,我们就把LSTM关于当前的记忆\({\tilde{c}}_t\)和长期的记忆\({c}_{t-1}\)组合在一起,形成了新的单元状态\({c}_t\)。由于遗忘门的控制,它可以保存很久很久之前的信息,由于输入门的控制,它又可以避免当前无关紧要的内容进入记忆。下面,我们要看看输出门,它控制了长期记忆对当前输出的影响:
\begin{equation}
	\label{eq:Lstm5}
	{o}_t=\sigma(W_o\cdot[{h}_{t-1},{x}_t]+{b}_o)
\end{equation}

图\ref{fig:Lstm9}表示输出门的计算:

\begin{figure}[!h]
	\centering
	\includegraphics[width=0.6\textwidth]{Lstm9.png}
	\caption{输出门}
	\label{fig:Lstm9}
\end{figure}

LSTM最终的输出,是由输出门和单元状态共同确定的:
\begin{equation}
	\label{eq:Lstm6}
	{h}_t={o}_t\circ \tanh({c}_t)
\end{equation}

图\ref{fig:Lstm10}表示LSTM最终输出的计算。

\begin{figure}[!h]
	\centering
	\includegraphics[width=0.6\textwidth]{Lstm10.png}
	\caption{LSTM最终输出}
	\label{fig:Lstm10}
\end{figure}

公式\ref{eq:Lstm1}到公式\ref{eq:Lstm6}就是LSTM前向计算的全部公式。至此,我们就把LSTM前向计算讲完了。

\section{长短时记忆网络的训练}\label{Lstm:3}

熟悉我们这个系列文章的同学都清楚,训练部分往往比前向计算部分复杂多了。
{LSTM} 的前向计算都这么复杂,那么,可想而知,它的训练算法一定是非常非常复杂的。
现在只有做几次深呼吸,再一头扎进公式海洋吧。

\subsection{LSTM训练算法框架}\label{Lstm:4}

LSTM的训练算法仍然是反向传播算法,对于这个算法,我们已经非常熟悉了。主要有下面三个步骤:

\begin{enumerate}
	\item
	      前向计算每个神经元的输出值,对于LSTM来说,即\({f}_t\)、\({i}_t\)、\({c}_t\)、\({o}_t\)、\({h}_t\)五个向量的值。计算方法已经在上一节中描述过了。
	\item
	      反向计算每个神经元的\textbf{误差项}\(\delta\)值。与\textbf{循环神经网络}一样,LSTM误差项的反向传播也是包括两个方向:一个是沿时间的反向传播,即从当前t时刻开始,计算每个时刻的误差项;一个是将误差项向上一层传播。
	\item
	      根据相应的误差项,计算每个权重的梯度。
\end{enumerate}


\subsection{关于公式和符号的说明}\label{Lstm:5}

首先,我们对推导中用到的一些公式、符号做一下必要的说明。

接下来的推导中,我们设定gate的激活函数为sigmoid函数,输出的激活函数为tanh函数。他们的导数分别为:
\begin{align*}
	\sigma(z)  & =y=\frac{1}{1+e^{-z}}            \\
	\sigma'(z) & =y(1-y)                          \\
	\tanh(z)   & =y=\frac{e^z-e^{-z}}{e^z+e^{-z}} \\
	\tanh'(z)  & =1-y^2
\end{align*}

从上面可以看出,sigmoid和tanh函数的导数都是原函数的函数。这样,我们一旦计算原函数的值,就可以用它来计算出导数的值。

LSTM需要学习的参数共有8组,分别是:遗忘门的权重矩阵\(W_f\)和偏置项\({b}_f\)、输入门的权重矩阵\(W_i\)和偏置项\({b}_i\)、输出门的权重矩阵\(W_o\)和偏置项\({b}_o\),以及计算单元状态的权重矩阵\(W_c\)和偏置项\({b}_c\)。因为权重矩阵的两部分在反向传播中使用不同的公式,因此在后续的推导中,权重矩阵\(W_f\)、\(W_i\)、\(W_c\)、\(W_o\)都将被写为分开的两个矩阵:\(W_{fh}\)、\(W_{fx}\)、\(W_{ih}\)、\(W_{ix}\)、\(W_{oh}\)、\(W_{ox}\)、\(W_{ch}\)、\(W_{cx}\)。

我们解释一下按元素乘\(\circ\)符号。当\(\circ\)作用于两个\textbf{向量}时,运算如下:
\[
	{a}\circ{b}=\begin{bmatrix}
		a_1 \\a_2\\a_3\\...\\a_n
	\end{bmatrix}\circ\begin{bmatrix}
		b_1 \\b_2\\b_3\\...\\b_n
	\end{bmatrix}=\begin{bmatrix}
		a_1b_1 \\a_2b_2\\a_3b_3\\...\\a_nb_n
	\end{bmatrix}
\]

当\(\circ\)作用于一个\textbf{向量}和一个\textbf{矩阵}时,运算如下:
\begin{align*}
	{a}\circ X & =\begin{bmatrix}
		a_1 \\a_2\\a_3\\...\\a_n
	\end{bmatrix}\circ\begin{bmatrix}
		x_{11} & x_{12} & x_{13} & ... & x_{1n} \\
		x_{21} & x_{22} & x_{23} & ... & x_{2n} \\
		x_{31} & x_{32} & x_{33} & ... & x_{3n} \\
		       &        & ...                   \\
		x_{n1} & x_{n2} & x_{n3} & ... & x_{nn} \\
	\end{bmatrix} \\
	           & =\begin{bmatrix}
		a_1x_{11} & a_1x_{12} & a_1x_{13} & ... & a_1x_{1n} \\
		a_2x_{21} & a_2x_{22} & a_2x_{23} & ... & a_2x_{2n} \\
		a_3x_{31} & a_3x_{32} & a_3x_{33} & ... & a_3x_{3n} \\
		          &           & ...                         \\
		a_nx_{n1} & a_nx_{n2} & a_nx_{n3} & ... & a_nx_{nn} \\
	\end{bmatrix}
\end{align*}

当\(\circ\)作用于两个\textbf{矩阵}时,两个矩阵对应位置的元素相乘。按元素乘可以在某些情况下简化矩阵和向量运算。例如,当一个对角矩阵右乘一个矩阵时,相当于用对角矩阵的对角线组成的向量按元素乘那个矩阵:
\[
	diag[{a}]X={a}\circ X
\]

当一个行向量右乘一个对角矩阵时,相当于这个行向量按元素乘那个矩阵对角线组成的向量:
\[
	{a}^Tdiag[{b}]={a}\circ{b}
\]

上面这两点,在我们后续推导中会多次用到。

在t时刻,LSTM的输出值为\({h}_t\)。我们定义t时刻的误差项\(\delta_t\)为:
\[
	\delta_t\overset{def}{=}\frac{\partial{E}}{\partial{{h}_t}}
\]

注意,和前面几篇文章不同,我们这里假设误差项是损失函数对输出值的导数,而不是对加权输入\(net_t^l\)的导数。因为LSTM有四个加权输入,分别对应\({f}_t\)、\({i}_t\)、\({c}_t\)、\({o}_t\),我们希望往上一层传递一个误差项而不是四个。但我们仍然需要定义出这四个加权输入,以及他们对应的误差项。

\begin{align*}
	{net}_{f,t}         & =W_f[{h}_{t-1},{x}_t]+{b}_f=W_{fh}{h}_{t-1}+W_{fx}{x}_t+{b}_f                                                                                                                                                                                                                                  \\
	{net}_{i,t}         & =W_i[{h}_{t-1},{x}_t]+{b}_i=W_{ih}{h}_{t-1}+W_{ix}{x}_t+{b}_i                                                                                                                                                                                                                                  \\
	{net}_{\tilde{c},t} & =W_c[{h}_{t-1},{x}_t]+{b}_c=W_{ch}{h}_{t-1}+W_{cx}{x}_t+{b}_c                                                                                                                                                                                                                                  \\
	{net}_{o,t}         & =W_o[{h}_{t-1},{x}_t]+{b}_o=W_{oh}{h}_{t-1}+W_{ox}{x}_t+{b}_o                                                                                                                                                                                                                                  \\
	\delta_{f,t}        & \overset{def}{=}\frac{\partial{E}}{\partial{{net}_{f,t}}}, \delta_{i,t}\overset{def}{=}\frac{\partial{E}}{\partial{{net}_{i,t}}}, \delta_{\tilde{c},t}\overset{def}{=}\frac{\partial{E}}{\partial{{net}_{\tilde{c},t}}}, \delta_{o,t}\overset{def}{=}\frac{\partial{E}}{\partial{{net}_{o,t}}}
\end{align*}

\subsection{误差项沿时间的反向传递}\label{Lstm:6}

沿时间反向传递误差项,就是要计算出t-1时刻的误差项\(\delta_{t-1}\)。
\begin{align*}
	\delta_{t-1}^T=\frac{\partial{E}}{\partial{{h_{t-1}}}}=\frac{\partial{E}}{\partial{{h_t}}}\frac{\partial{{h_t}}}{\partial{{h_{t-1}}}}=\delta_{t}^T\frac{\partial{{h_t}}}{\partial{{h_{t-1}}}}
\end{align*}

我们知道,\(\frac{\partial{{h_t}}}{\partial{{h_{t-1}}}}\)是一个Jacobian矩阵。如果隐藏层h的维度是N的话,那么它就是一个\(N\times N\)矩阵。为了求出它,我们列出\({h}_t\)的计算公式,即前面的公式\ref{eq:Lstm4}和公式\ref{eq:Lstm6}:
\begin{align*}
	{h}_t & ={o}_t\circ \tanh({c}_t)                     \\
	{c}_t & ={f}_t\circ{c}_{t-1}+{i}_t\circ{\tilde{c}}_t
\end{align*}

显然,\({o}_t\)、\({f}_t\)、\({i}_t\)、\({\tilde{c}}_t\)都是\({h}_{t-1}\)的函数,那么,利用全导数公式可得:
\begin{align}
	\delta_t^T\frac{\partial{{h_t}}}{\partial{{h_{t-1}}}} & =\delta_t^T\frac{\partial{{h_t}}}{\partial{{o}_t}}\frac{\partial{{o}_t}}{\partial{{net}_{o,t}}}\frac{\partial{{net}_{o,t}}}{\partial{{h_{t-1}}}}
	+\delta_t^T\frac{\partial{{h_t}}}{\partial{{c}_t}}\frac{\partial{{c}_t}}{\partial{{f_{t}}}}\frac{\partial{{f}_t}}{\partial{{net}_{f,t}}}\frac{\partial{{net}_{f,t}}}{\partial{{h_{t-1}}}}\notag                                                   \\
	                                                      & +\delta_t^T\frac{\partial{{h_t}}}{\partial{{c}_t}}\frac{\partial{{c}_t}}{\partial{{i_{t}}}}\frac{\partial{{i}_t}}{\partial{{net}_{i,t}}}\frac{\partial{{net}_{i,t}}}{\partial{{h_{t-1}}}}
	+\delta_t^T\frac{\partial{{h_t}}}{\partial{{c}_t}}\frac{\partial{{c}_t}}{\partial{{\tilde{c}}_{t}}}\frac{\partial{{\tilde{c}}_t}}{\partial{{net}_{\tilde{c},t}}}\frac{\partial{{net}_{\tilde{c},t}}}{\partial{{h_{t-1}}}}\notag                   \\
	                                                      & =\delta_{o,t}^T\frac{\partial{{net}_{o,t}}}{\partial{{h_{t-1}}}}
	+\delta_{f,t}^T\frac{\partial{{net}_{f,t}}}{\partial{{h_{t-1}}}}
	+\delta_{i,t}^T\frac{\partial{{net}_{i,t}}}{\partial{{h_{t-1}}}}
	+\delta_{\tilde{c},t}^T\frac{\partial{{net}_{\tilde{c},t}}}{\partial{{h_{t-1}}}}\label{eq:Lstm7}
\end{align}


下面,我们要把公式\ref{eq:Lstm7}中的每个偏导数都求出来。根据公式\ref{eq:Lstm6},我们可以求出:
\begin{align*}
	\frac{\partial{{h_t}}}{\partial{{o}_t}} & =diag[\tanh({c}_t)]                 \\
	\frac{\partial{{h_t}}}{\partial{{c}_t}} & =diag[{o}_t\circ(1-\tanh({c}_t)^2)]
\end{align*}

根据公式\ref{eq:Lstm4},我们可以求出:
\begin{align*}
	\frac{\partial{{c}_t}}{\partial{{f_{t}}}}=diag[{c}_{t-1}],\quad
	\frac{\partial{{c}_t}}{\partial{{i_{t}}}}=diag[{\tilde{c}}_t],\quad
	\frac{\partial{{c}_t}}{\partial{{\tilde{c}_{t}}}}=diag[{i}_t]
\end{align*}

因为:
\begin{align*}
	{o}_t         & =\sigma({net}_{o,t}),\quad {net}_{o,t}=W_{oh}{h}_{t-1}+W_{ox}{x}_t+{b}_o                \\
	{f}_t         & =\sigma({net}_{f,t}),\quad {net}_{f,t}=W_{fh}{h}_{t-1}+W_{fx}{x}_t+{b}_f                \\
	{i}_t         & =\sigma({net}_{i,t}),\quad {net}_{i,t}=W_{ih}{h}_{t-1}+W_{ix}{x}_t+{b}_i                \\
	{\tilde{c}}_t & =\tanh({net}_{\tilde{c},t}),\quad {net}_{\tilde{c},t}=W_{ch}{h}_{t-1}+W_{cx}{x}_t+{b}_c
\end{align*}

我们很容易得出:
\begin{align*}
	\frac{\partial{{o}_t}}{\partial{{net}_{o,t}}}                 & =diag[{o}_t\circ(1-{o}_t)],\quad \frac{\partial{{net}_{o,t}}}{\partial{{h_{t-1}}}}=W_{oh}       \\
	\frac{\partial{{f}_t}}{\partial{{net}_{f,t}}}                 & =diag[{f}_t\circ(1-{f}_t)],\quad \frac{\partial{{net}_{f,t}}}{\partial{{h}_{t-1}}}=W_{fh}       \\
	\frac{\partial{{i}_t}}{\partial{{net}_{i,t}}}                 & =diag[{i}_t\circ(1-{i}_t)],\quad \frac{\partial{{net}_{i,t}}}{\partial{{h}_{t-1}}}=W_{ih}       \\
	\frac{\partial{{\tilde{c}}_t}}{\partial{{net}_{\tilde{c},t}}} & =diag[1-{\tilde{c}}_t^2],\quad \frac{\partial{{net}_{\tilde{c},t}}}{\partial{{h}_{t-1}}}=W_{ch}
\end{align*}

将上述偏导数带入到公式\ref{eq:Lstm7},我们得到:
\begin{align}
	\delta_{t-1} & =\delta_{o,t}^T\frac{\partial{{net}_{o,t}}}{\partial{{h_{t-1}}}}
	+\delta_{f,t}^T\frac{\partial{{net}_{f,t}}}{\partial{{h_{t-1}}}}
	+\delta_{i,t}^T\frac{\partial{{net}_{i,t}}}{\partial{{h_{t-1}}}}
	+\delta_{\tilde{c},t}^T\frac{\partial{{net}_{\tilde{c},t}}}{\partial{{h_{t-1}}}}\notag \\
	             & =\delta_{o,t}^T W_{oh}
	+\delta_{f,t}^TW_{fh}
	+\delta_{i,t}^TW_{ih}
	+\delta_{\tilde{c},t}^TW_{ch}\label{eq:Lstm8}
\end{align}

根据\(\delta_{o,t}\)、\(\delta_{f,t}\)、\(\delta_{i,t}\)、\(\delta_{\tilde{c},t}\)的定义,可知:
\begin{align}
	\delta_{o,t}^T         & =\delta_t^T\circ\tanh({c}_t)\circ{o}_t\circ(1-{o}_t)\label{eq:Lstm9}                                    \\
	\delta_{f,t}^T         & =\delta_t^T\circ{o}_t\circ(1-\tanh({c}_t)^2)\circ{c}_{t-1}\circ{f}_t\circ(1-{f}_t)\label{eq:Lstm10}     \\
	\delta_{i,t}^T         & =\delta_t^T\circ{o}_t\circ(1-\tanh({c}_t)^2)\circ{\tilde{c}}_t\circ{i}_t\circ(1-{i}_t)\label{eq:Lstm11} \\
	\delta_{\tilde{c},t}^T & =\delta_t^T\circ{o}_t\circ(1-\tanh({c}_t)^2)\circ{i}_t\circ(1-{\tilde{c}}^2)\label{eq:Lstm12}
\end{align}

公式\ref{eq:Lstm8}到公式\ref{eq:Lstm12}就是将误差沿时间反向传播一个时刻的公式。有了它,我们可以写出将误差项向前传递到任意k时刻的公式:
\begin{equation}
	\label{eq:Lstm13}
	\delta_k^T=\prod_{j=k}^{t-1}\delta_{o,j}^TW_{oh}
	+\delta_{f,j}^TW_{fh}
	+\delta_{i,j}^TW_{ih}
	+\delta_{\tilde{c},j}^TW_{ch}
\end{equation}

\subsection{将误差项传递到上一层}\label{Lstm:7}

我们假设当前为第l层,定义l-1层的误差项是误差函数对l-1层\textbf{加权输入}的导数,即:
\[
	\delta_t^{l-1}\overset{def}{=}\frac{\partial{E}}{{net}_t^{l-1}}
\]

本次LSTM的输入\(x_t\)由下面的公式计算:
\[
	{x}_t^l=f^{l-1}({net}_t^{l-1})
\]
上式中,\(f^{l-1}\)表示第l-1层的\textbf{激活函数}。

因为\({net}_{f,t}^l\)、\({net}_{i,t}^l\)、\({net}_{\tilde{c},t}^l\)、\({net}_{o,t}^l\)都是\({x}_t\)的函数,\({x}_t\)又是\({net}_t^{l-1}\)的函数,因此,要求出E对\({net}_t^{l-1}\)的导数,就需要使用全导数公式:

\begin{align}
	\frac{\partial{E}}{\partial{{net}_t^{l-1}}} & =\frac{\partial{E}}{\partial{{{net}_{f,t}^l}}}\frac{\partial{{{net}_{f,t}^l}}}{\partial{{x}_t^l}}\frac{\partial{{x}_t^l}}{\partial{{{net}_t^{l-1}}}}
	+\frac{\partial{E}}{\partial{{{net}_{i,t}^l}}}\frac{\partial{{{net}_{i,t}^l}}}{\partial{{x}_t^l}}\frac{\partial{{x}_t^l}}{\partial{{{net}_t^{l-1}}}}
	\notag                                                                                                                                                                                                                                       \\&+\frac{\partial{E}}{\partial{{{net}_{\tilde{c},t}^l}}}\frac{\partial{{{net}_{\tilde{c},t}^l}}}{\partial{{x}_t^l}}\frac{\partial{{x}_t^l}}{\partial{{{net}_t^{l-1}}}}
	+\frac{\partial{E}}{\partial{{{net}_{o,t}^l}}}\frac{\partial{{{net}_{o,t}^l}}}{\partial{{x}_t^l}}\frac{\partial{{x}_t^l}}{\partial{{{net}_t^{l-1}}}}\notag                                                                                   \\
	                                            & =\delta_{f,t}^TW_{fx}\circ f'({net}_t^{l-1})+\delta_{i,t}^TW_{ix}\circ f'({net}_t^{l-1})+\delta_{\tilde{c},t}^TW_{cx}\circ f'({net}_t^{l-1})+\delta_{o,t}^TW_{ox}\circ f'({net}_t^{l-1})\notag \\
	                                            & =(\delta_{f,t}^TW_{fx}+\delta_{i,t}^TW_{ix}+\delta_{\tilde{c},t}^TW_{cx}+\delta_{o,t}^TW_{ox})\circ f'({net}_t^{l-1})\label{eq:Lstm14}
\end{align}

公式\ref{eq:Lstm14}就是将误差传递到上一层的公式。



\subsection{权重梯度的计算}\label{Lstm:8}

对于\(W_{fh}\)、\(W_{ih}\)、\(W_{ch}\)、\(W_{oh}\)的权重梯度,我们知道它的梯度是各个时刻梯度之和(证明过程请参考第\ref{chap:Rnn}循环神经网络),我们首先求出它们在t时刻的梯度,然后再求出他们最终的梯度。

我们已经求得了误差项\(\delta_{o,t}\)、\(\delta_{f,t}\)、\(\delta_{i,t}\)、\(\delta_{\tilde{c},t}\),很容易求出t时刻的\(W_{oh}\)、的\(W_{ih}\)、的\(W_{fh}\)、的\(W_{ch}\):
\begin{align*}
	\frac{\partial{E}}{\partial{W_{oh,t}}} & =\frac{\partial{E}}{\partial{{net}_{o,t}}}\frac{\partial{{net}_{o,t}}}{\partial{W_{oh,t}}}=\delta_{o,t}{h}_{t-1}^T                         \\
	\frac{\partial{E}}{\partial{W_{fh,t}}} & =\frac{\partial{E}}{\partial{{net}_{f,t}}}\frac{\partial{{net}_{f,t}}}{\partial{W_{fh,t}}}=\delta_{f,t}{h}_{t-1}^T                         \\
	\frac{\partial{E}}{\partial{W_{ih,t}}} & =\frac{\partial{E}}{\partial{{net}_{i,t}}}\frac{\partial{{net}_{i,t}}}{\partial{W_{ih,t}}}=\delta_{i,t}{h}_{t-1}^T                         \\
	\frac{\partial{E}}{\partial{W_{ch,t}}} & =\frac{\partial{E}}{\partial{{net}_{\tilde{c},t}}}\frac{\partial{{net}_{\tilde{c},t}}}{\partial{W_{ch,t}}}=\delta_{\tilde{c},t}{h}_{t-1}^T
\end{align*}

将各个时刻的梯度加在一起,就能得到最终的梯度:
\begin{align*}
	\frac{\partial{E}}{\partial{W_{oh}}} & =\sum_{j=1}^t\delta_{o,j}{h}_{j-1}^T         \\
	\frac{\partial{E}}{\partial{W_{fh}}} & =\sum_{j=1}^t\delta_{f,j}{h}_{j-1}^T         \\
	\frac{\partial{E}}{\partial{W_{ih}}} & =\sum_{j=1}^t\delta_{i,j}{h}_{j-1}^T         \\
	\frac{\partial{E}}{\partial{W_{ch}}} & =\sum_{j=1}^t\delta_{\tilde{c},j}{h}_{j-1}^T
\end{align*}

对于偏置项\({b}_f\)、\({b}_i\)、\({b}_c\)、\({b}_o\)的梯度,也是将各个时刻的梯度加在一起。下面是各个时刻的偏置项梯度:
\begin{align*}
	\frac{\partial{E}}{\partial{{b}_{o,t}}} & =\frac{\partial{E}}{\partial{{net}_{o,t}}}\frac{\partial{{net}_{o,t}}}{\partial{{b}_{o,t}}}=\delta_{o,t}                         \\
	\frac{\partial{E}}{\partial{{b}_{f,t}}} & =\frac{\partial{E}}{\partial{{net}_{f,t}}}\frac{\partial{{net}_{f,t}}}{\partial{{b}_{f,t}}}=\delta_{f,t}                         \\
	\frac{\partial{E}}{\partial{{b}_{i,t}}} & =\frac{\partial{E}}{\partial{{net}_{i,t}}}\frac{\partial{{net}_{i,t}}}{\partial{{b}_{i,t}}}=\delta_{i,t}                         \\
	\frac{\partial{E}}{\partial{{b}_{c,t}}} & =\frac{\partial{E}}{\partial{{net}_{\tilde{c},t}}}\frac{\partial{{net}_{\tilde{c},t}}}{\partial{{b}_{c,t}}}=\delta_{\tilde{c},t}
\end{align*}

下面是最终的偏置项梯度,即将各个时刻的偏置项梯度加在一起:
\begin{align*}
	\frac{\partial{E}}{\partial{{b}_o}}=\sum_{j=1}^t\delta_{o,j},
	\frac{\partial{E}}{\partial{{b}_i}}=\sum_{j=1}^t\delta_{i,j},
	\frac{\partial{E}}{\partial{{b}_f}}=\sum_{j=1}^t\delta_{f,j},
	\frac{\partial{E}}{\partial{{b}_c}}=\sum_{j=1}^t\delta_{\tilde{c},j}
\end{align*}

对于\(W_{fx}\)、\(W_{ix}\)、\(W_{cx}\)、\(W_{ox}\)的权重梯度,只需要根据相应的误差项直接计算即可:
\begin{align*}
	\frac{\partial{E}}{\partial{W_{ox}}} & =\frac{\partial{E}}{\partial{{net}_{o,t}}}\frac{\partial{{net}_{o,t}}}{\partial{W_{ox}}}=\delta_{o,t}{x}_{t}^T                         \\
	\frac{\partial{E}}{\partial{W_{fx}}} & =\frac{\partial{E}}{\partial{{net}_{f,t}}}\frac{\partial{{net}_{f,t}}}{\partial{W_{fx}}}=\delta_{f,t}{x}_{t}^T                         \\
	\frac{\partial{E}}{\partial{W_{ix}}} & =\frac{\partial{E}}{\partial{{net}_{i,t}}}\frac{\partial{{net}_{i,t}}}{\partial{W_{ix}}}=\delta_{i,t}{x}_{t}^T                         \\
	\frac{\partial{E}}{\partial{W_{cx}}} & =\frac{\partial{E}}{\partial{{net}_{\tilde{c},t}}}\frac{\partial{{net}_{\tilde{c},t}}}{\partial{W_{cx}}}=\delta_{\tilde{c},t}{x}_{t}^T
\end{align*}

以上就是LSTM的训练算法的全部公式。因为这里面存在很多重复的模式,仔细看看,会发觉并不是太复杂。

当然,LSTM存在着相当多的变体,读者可以在互联网上找到很多资料。因为大家已经熟悉了基本LSTM的算法,因此理解这些变体比较容易,因此本文就不再赘述了。




\section{编程实战:长短时记忆网络的实现}\label{Lstm:9}

\begin{note}
	完整代码请参考GitHub: \url{https://github.com/hanbt/learn_dl/blob/master/lstm.py}
	(python2.7)
\end{note}

在下面的实现中,LSTMLayer的参数包括输入维度、输出维度、隐藏层维度,单元状态维度等于隐藏层维度。gate的激活函数为sigmoid函数,输出的激活函数为tanh。

\subsection{激活函数的实现}\label{Lstm:10}
我们先实现两个激活函数:sigmoid和tanh。
\begin{lstlisting}
class SigmoidActivator(object):
    def forward(self, weighted_input):
        return 1.0 / (1.0 + np.exp(-weighted_input))
    def backward(self, output):
        return output * (1 - output)
class TanhActivator(object):
    def forward(self, weighted_input):
        return 2.0 / (1.0 + np.exp(-2 * weighted_input)) - 1.0
    def backward(self, output):
        return 1 - output * output
\end{lstlisting}


\subsection{LSTM初始化}\label{Lstm:11}

和前两篇文章代码架构一样,我们把LSTM的实现放在LstmLayer类中。

根据LSTM前向计算和方向传播算法,我们需要初始化一系列矩阵和向量。这些矩阵和向量有两类用途,一类是用于保存模型参数,例如\(W_f\)、\(W_i\)、\(W_o\)、\(W_c\)、\({b}_f\)、\({b}_i\)、\({b}_o\)、\({b}_c\);另一类是保存各种中间计算结果,以便于反向传播算法使用,它们包括\({h}_t\)、\({f}_t\)、\({i}_t\)、\({o}_t\)、\({c}_t\)、\({\tilde{c}}_t\)、\(\delta_t\)、\(\delta_{f,t}\)、\(\delta_{i,t}\)、\(\delta_{o,t}\)、\(\delta_{\tilde{c},t}\),以及各个权重对应的梯度。

在构造函数的初始化中,只初始化了与forward计算相关的变量,与backward相关的变量没有初始化。这是因为构造LSTM对象的时候,我们还不知道它未来是用于训练(既有forward又有backward)还是推理(只有forward)。
\begin{lstlisting}
class LstmLayer(object):
    def __init__(self, input_width, state_width, 
                 learning_rate):
        self.input_width = input_width
        self.state_width = state_width
        self.learning_rate = learning_rate
        # 门的激活函数
        self.gate_activator = SigmoidActivator()
        # 输出的激活函数
        self.output_activator = TanhActivator()
        # 当前时刻初始化为t0
        self.times = 0       
        # 各个时刻的单元状态向量c
        self.c_list = self.init_state_vec()
        # 各个时刻的输出向量h
        self.h_list = self.init_state_vec()
        # 各个时刻的遗忘门f
        self.f_list = self.init_state_vec()
        # 各个时刻的输入门i
        self.i_list = self.init_state_vec()
        # 各个时刻的输出门o
        self.o_list = self.init_state_vec()
        # 各个时刻的即时状态c~
        self.ct_list = self.init_state_vec()
        # 遗忘门权重矩阵Wfh, Wfx, 偏置项bf
        self.Wfh, self.Wfx, self.bf = (
            self.init_weight_mat())
        # 输入门权重矩阵Wfh, Wfx, 偏置项bf
        self.Wih, self.Wix, self.bi = (
            self.init_weight_mat())
        # 输出门权重矩阵Wfh, Wfx, 偏置项bf
        self.Woh, self.Wox, self.bo = (
            self.init_weight_mat())
        # 单元状态权重矩阵Wfh, Wfx, 偏置项bf
        self.Wch, self.Wcx, self.bc = (
            self.init_weight_mat())
    def init_state_vec(self):
        '''
        初始化保存状态的向量
        '''
        state_vec_list = []
        state_vec_list.append(np.zeros(
            (self.state_width, 1)))
        return state_vec_list
    def init_weight_mat(self):
        '''
        初始化权重矩阵
        '''
        Wh = np.random.uniform(-1e-4, 1e-4,
            (self.state_width, self.state_width))
        Wx = np.random.uniform(-1e-4, 1e-4,
            (self.state_width, self.input_width))
        b = np.zeros((self.state_width, 1))
        return Wh, Wx, b
\end{lstlisting}

\subsection{前向计算的实现}\label{Lstm:12}

forward方法实现了LSTM的前向计算:
\begin{lstlisting}
    def forward(self, x):
        '''
        根据式1-式6进行前向计算
        '''
        self.times += 1
        # 遗忘门
        fg = self.calc_gate(x, self.Wfx, self.Wfh, 
            self.bf, self.gate_activator)
        self.f_list.append(fg)
        # 输入门
        ig = self.calc_gate(x, self.Wix, self.Wih,
            self.bi, self.gate_activator)
        self.i_list.append(ig)
        # 输出门
        og = self.calc_gate(x, self.Wox, self.Woh,
            self.bo, self.gate_activator)
        self.o_list.append(og)
        # 即时状态
        ct = self.calc_gate(x, self.Wcx, self.Wch,
            self.bc, self.output_activator)
        self.ct_list.append(ct)
        # 单元状态
        c = fg * self.c_list[self.times - 1] + ig * ct
        self.c_list.append(c)
        # 输出
        h = og * self.output_activator.forward(c)
        self.h_list.append(h)
    def calc_gate(self, x, Wx, Wh, b, activator):
        '''
        计算门
        '''
        h = self.h_list[self.times - 1] # 上次的LSTM输出
        net = np.dot(Wh, h) + np.dot(Wx, x) + b
        gate = activator.forward(net)
        return gate
\end{lstlisting}

从上面的代码我们可以看到,门的计算都是相同的算法,而门和\({\tilde{c}_t}\)的计算仅仅是激活函数不同。因此我们提出了calc\_gate方法,这样减少了很多重复代码。


\subsection{反向传播算法的实现}\label{Lstm:13}

backward方法实现了LSTM的反向传播算法。需要注意的是,与backword相关的内部状态变量是在调用backward方法之后才初始化的。这种延迟初始化的一个好处是,如果LSTM只是用来推理,那么就不需要初始化这些变量,节省了很多内存。
\begin{lstlisting}
    def backward(self, x, delta_h, activator):
        '''
        实现LSTM训练算法
        '''
        self.calc_delta(delta_h, activator)
        self.calc_gradient(x)
\end{lstlisting}

算法主要分成两个部分,一部分使计算误差项:
\begin{lstlisting}
    def calc_delta(self, delta_h, activator):
        # 初始化各个时刻的误差项
        self.delta_h_list = self.init_delta()  # 输出误差项
        self.delta_o_list = self.init_delta()  # 输出门误差项
        self.delta_i_list = self.init_delta()  # 输入门误差项
        self.delta_f_list = self.init_delta()  # 遗忘门误差项
        self.delta_ct_list = self.init_delta() # 即时输出误差项
        # 保存从上一层传递下来的当前时刻的误差项
        self.delta_h_list[-1] = delta_h
        # 迭代计算每个时刻的误差项
        for k in range(self.times, 0, -1):
            self.calc_delta_k(k)
    def init_delta(self):
        '''
        初始化误差项
        '''
        delta_list = []
        for i in range(self.times + 1):
            delta_list.append(np.zeros(
                (self.state_width, 1)))
        return delta_list
    def calc_delta_k(self, k):
        '''
        根据k时刻的delta_h,计算k时刻的delta_f、
        delta_i、delta_o、delta_ct,以及k-1时刻的delta_h
        '''
        # 获得k时刻前向计算的值
        ig = self.i_list[k]
        og = self.o_list[k]
        fg = self.f_list[k]
        ct = self.ct_list[k]
        c = self.c_list[k]
        c_prev = self.c_list[k-1]
        tanh_c = self.output_activator.forward(c)
        delta_k = self.delta_h_list[k]
        # 根据式9计算delta_o
        delta_o = (delta_k * tanh_c * 
            self.gate_activator.backward(og))
        delta_f = (delta_k * og * 
            (1 - tanh_c * tanh_c) * c_prev *
            self.gate_activator.backward(fg))
        delta_i = (delta_k * og * 
            (1 - tanh_c * tanh_c) * ct *
            self.gate_activator.backward(ig))
        delta_ct = (delta_k * og * 
            (1 - tanh_c * tanh_c) * ig *
            self.output_activator.backward(ct))
        delta_h_prev = (
                np.dot(delta_o.transpose(), self.Woh) +
                np.dot(delta_i.transpose(), self.Wih) +
                np.dot(delta_f.transpose(), self.Wfh) +
                np.dot(delta_ct.transpose(), self.Wch)
            ).transpose()
        # 保存全部delta值
        self.delta_h_list[k-1] = delta_h_prev
        self.delta_f_list[k] = delta_f
        self.delta_i_list[k] = delta_i
        self.delta_o_list[k] = delta_o
        self.delta_ct_list[k] = delta_ct
\end{lstlisting}

另一部分是计算梯度:
\begin{lstlisting}
    def calc_gradient(self, x):
        # 初始化遗忘门权重梯度矩阵和偏置项
        self.Wfh_grad, self.Wfx_grad, self.bf_grad = (
            self.init_weight_gradient_mat())
        # 初始化输入门权重梯度矩阵和偏置项
        self.Wih_grad, self.Wix_grad, self.bi_grad = (
            self.init_weight_gradient_mat())
        # 初始化输出门权重梯度矩阵和偏置项
        self.Woh_grad, self.Wox_grad, self.bo_grad = (
            self.init_weight_gradient_mat())
        # 初始化单元状态权重梯度矩阵和偏置项
        self.Wch_grad, self.Wcx_grad, self.bc_grad = (
            self.init_weight_gradient_mat())
       # 计算对上一次输出h的权重梯度
        for t in range(self.times, 0, -1):
            # 计算各个时刻的梯度
            (Wfh_grad, bf_grad,
            Wih_grad, bi_grad,
            Woh_grad, bo_grad,
            Wch_grad, bc_grad) = (
                self.calc_gradient_t(t))
            # 实际梯度是各时刻梯度之和
            self.Wfh_grad += Wfh_grad
            self.bf_grad += bf_grad
            self.Wih_grad += Wih_grad
            self.bi_grad += bi_grad
            self.Woh_grad += Woh_grad
            self.bo_grad += bo_grad
            self.Wch_grad += Wch_grad
            self.bc_grad += bc_grad
            print '-----%d-----' % t
            print Wfh_grad
            print self.Wfh_grad
        # 计算对本次输入x的权重梯度
        xt = x.transpose()
        self.Wfx_grad = np.dot(self.delta_f_list[-1], xt)
        self.Wix_grad = np.dot(self.delta_i_list[-1], xt)
        self.Wox_grad = np.dot(self.delta_o_list[-1], xt)
        self.Wcx_grad = np.dot(self.delta_ct_list[-1], xt)
    def init_weight_gradient_mat(self):
        '''
        初始化权重矩阵
        '''
        Wh_grad = np.zeros((self.state_width,
            self.state_width))
        Wx_grad = np.zeros((self.state_width,
            self.input_width))
        b_grad = np.zeros((self.state_width, 1))
        return Wh_grad, Wx_grad, b_grad
    def calc_gradient_t(self, t):
        '''
        计算每个时刻t权重的梯度
        '''
        h_prev = self.h_list[t-1].transpose()
        Wfh_grad = np.dot(self.delta_f_list[t], h_prev)
        bf_grad = self.delta_f_list[t]
        Wih_grad = np.dot(self.delta_i_list[t], h_prev)
        bi_grad = self.delta_f_list[t]
        Woh_grad = np.dot(self.delta_o_list[t], h_prev)
        bo_grad = self.delta_f_list[t]
        Wch_grad = np.dot(self.delta_ct_list[t], h_prev)
        bc_grad = self.delta_ct_list[t]
        return Wfh_grad, bf_grad, Wih_grad, bi_grad, \
               Woh_grad, bo_grad, Wch_grad, bc_grad
\end{lstlisting}

\subsection{梯度下降算法的实现}\label{Lstm:14}
下面是用梯度下降算法来更新权重:
\begin{lstlisting}
    def update(self):
    '''
    按照梯度下降,更新权重
    '''
    self.Wfh -= self.learning_rate * self.Whf_grad
    self.Wfx -= self.learning_rate * self.Whx_grad
    self.bf -= self.learning_rate * self.bf_grad
    self.Wih -= self.learning_rate * self.Whi_grad
    self.Wix -= self.learning_rate * self.Whi_grad
    self.bi -= self.learning_rate * self.bi_grad
    self.Woh -= self.learning_rate * self.Wof_grad
    self.Wox -= self.learning_rate * self.Wox_grad
    self.bo -= self.learning_rate * self.bo_grad
    self.Wch -= self.learning_rate * self.Wcf_grad
    self.Wcx -= self.learning_rate * self.Wcx_grad
    self.bc -= self.learning_rate * self.bc_grad
\end{lstlisting}





\subsection{梯度检查的实现}\label{Lstm:15}

和RecurrentLayer一样,为了支持梯度检查,我们需要支持重置内部状态:
\begin{lstlisting}
    def reset_state(self):
        # 当前时刻初始化为t0
        self.times = 0       
        # 各个时刻的单元状态向量c
        self.c_list = self.init_state_vec()
        # 各个时刻的输出向量h
        self.h_list = self.init_state_vec()
        # 各个时刻的遗忘门f
        self.f_list = self.init_state_vec()
        # 各个时刻的输入门i
        self.i_list = self.init_state_vec()
        # 各个时刻的输出门o
        self.o_list = self.init_state_vec()
        # 各个时刻的即时状态c~
        self.ct_list = self.init_state_vec()
\end{lstlisting}

最后,是梯度检查的代码:
\begin{lstlisting}
    def data_set():
    x = [np.array([[1], [2], [3]]),
         np.array([[2], [3], [4]])]
    d = np.array([[1], [2]])
    return x, d
def gradient_check():
    '''
    梯度检查
    '''
    # 设计一个误差函数,取所有节点输出项之和
    error_function = lambda o: o.sum()
    lstm = LstmLayer(3, 2, 1e-3)
    # 计算forward值
    x, d = data_set()
    lstm.forward(x[0])
    lstm.forward(x[1])
    # 求取sensitivity map
    sensitivity_array = np.ones(lstm.h_list[-1].shape,
                                dtype=np.float64)
    # 计算梯度
    lstm.backward(x[1], sensitivity_array, IdentityActivator())
    # 检查梯度
    epsilon = 10e-4
    for i in range(lstm.Wfh.shape[0]):
        for j in range(lstm.Wfh.shape[1]):
            lstm.Wfh[i,j] += epsilon
            lstm.reset_state()
            lstm.forward(x[0])
            lstm.forward(x[1])
            err1 = error_function(lstm.h_list[-1])
            lstm.Wfh[i,j] -= 2*epsilon
            lstm.reset_state()
            lstm.forward(x[0])
            lstm.forward(x[1])
            err2 = error_function(lstm.h_list[-1])
            expect_grad = (err1 - err2) / (2 * epsilon)
            lstm.Wfh[i,j] += epsilon
            print 'weights(%d,%d): expected - actural %.4e - %.4e' % (
                i, j, expect_grad, lstm.Wfh_grad[i,j])
    return lstm
\end{lstlisting}


我们只对\(W_{fh}\)做了检查,读者可以自行增加对其他梯度的检查。下面是某次梯度检查的结果:

\includegraphics[width=0.8\textwidth]{Lstm11.png}


\section{GRU}\label{Lstm:16}

前面我们讲了一种普通的LSTM,事实上LSTM存在很多\textbf{变体},许多论文中的LSTM都或多或少的不太一样。在众多的LSTM变体中,\textbf{GRU (Gated Recurrent Unit)}也许是最成功的一种。它对LSTM做了很多简化,同时却保持着和LSTM相同的效果。因此,GRU最近变得越来越流行。

GRU对LSTM做了两个大改动:

\begin{enumerate}
	\item
	      将输入门、遗忘门、输出门变为两个门:更新门(Update Gate)\({z}_t\)和重置门(Reset Gate)\({r}_t\)。
	\item
	      将单元状态与输出合并为一个状态:\({h}\)。
\end{enumerate}

GRU的前向计算公式为:
\begin{align*}
	{z}_t         & =\sigma(W_z\cdot[{h}_{t-1},{x}_t])               \\
	{r}_t         & =\sigma(W_r\cdot[{h}_{t-1},{x}_t])               \\
	{\tilde{h}}_t & =\tanh(W\cdot[{r}_t\circ{h}_{t-1},{x}_t])        \\
	{h}           & =(1-{z}_t)\circ{h}_{t-1}+{z}_t\circ{\tilde{h}}_t
\end{align*}


图\ref{fig:Lstm12}是GRU的示意图:

\begin{figure}[!h]
	\centering
	\includegraphics[width=0.6\textwidth]{Lstm12.png}
	\caption{GRU}
	\label{fig:Lstm12}
\end{figure}

GRU的训练算法比LSTM简单一些,留给读者自行推导,本文就不再赘述了。

\section{小结}

至此,LSTM---也许是结构最复杂的一类神经网络---就讲完了,相信拿下前几篇文章的读者们搞定这篇文章也不在话下吧!现在我们已经了解\textbf{循环神经网络}和它最流行的变体---\textbf{LSTM},它们都可以用来处理序列。但是,有时候仅仅拥有处理序列的能力还不够,还需要处理比序列更为复杂的结构(比如树结构),这时候就需要用到另外一类网络:\textbf{递归神经网络(Recursive Neural Network)},巧合的是,它的缩写也是\textbf{RNN}。在下一篇文章中,我们将介绍\textbf{递归神经网络}和它的训练算法。现在,漫长的烧脑暂告一段落,休息一下吧:)







\input{./chapter/recu.tex}
% \chapter{temp}



\begin{figure}[htbp]
    \centering
    \includegraphics[width=0.7\textwidth]{Cnn1.png}
    \caption{Relu函数}
    \label{fig:Cnn1}
\end{figure}

\begin{figure}[htbp]
    \centering
    \includegraphics[width=1\textwidth]{Rnn16.png}
    \caption{cross correlation}
    \label{fig:Rnn16}
\end{figure}

\begin{figure}[!h]
    \centering
    \includegraphics[width=1\textwidth]{Recu1.png}
    \caption{cross correlation}
    \label{fig:Recu1}
\end{figure}

\begin{lstlisting}[numbers=none]
    INPUT -> [[CONV]*N -> POOL?]*M -> [FC]*K
\end{lstlisting}


\begin{note}
    ???????GitHub: \url{https://github.com/hanbt/learn_dl/blob/master/rnn.py}
    (python2.7)
\end{note}


\begin{lstlisting}

\end{lstlisting}














?\ref{fig:Recu1}

??\ref{eq:Recu1}


\\\[
(.+?)\\qquad\(?(1)\)
\\\]


\begin{equation}
    \label{eq:Recu}
\end{equation}


\end{document}
